\documentclass[a4paper]{book}

\usepackage[utf8]{inputenc}
\usepackage[greek,spanish,es-tabla,es-nodecimaldot,es-noindentfirst]{babel}
\usepackage{babelbib}
\usepackage{nccmath}
\usepackage[oldvoltagedirection]{circuitikz}
\usepackage{amsthm}
\usepackage{lipsum}
\usepackage{tcolorbox}
\usepackage[thicklines]{cancel}
\usepackage{mathtools}
\usepackage{amssymb}
\usepackage{amsmath}
\usepackage{caption}
\usepackage{subcaption}   
\usepackage{color}       
\usepackage{verbatim}     
\usepackage{enumerate}
\usepackage{geometry} 
\geometry{a4paper,left=35mm,right=35mm,top=15mm,bottom=15mm}
\usepackage{isotope}
\usepackage{maybemath} 
\usepackage{upgreek}
\usepackage{wasysym} 
\usepackage[italic]{hepparticles}
\usepackage{subdepth}
\usepackage{physics}
\usepackage{braket}
\usepackage{tensor}
\usepackage{chemformula} 
\usepackage{tikz} \usetikzlibrary{babel}
\usepackage{siunitx}
\usepackage{url}
\usepackage{multirow}
\usepackage{multicol}
\usepackage[colorlinks=true]{hyperref}
\hypersetup{
citecolor = blue,
linkcolor = blue,
urlcolor = blue,
pdfauthor = {Javier Rodrigo López}
}
\usepackage{eso-pic}
\usepackage{siunitx}
\sisetup{
round-mode      = places,
round-precision = 2,
}

% Títulos
\usepackage{titlesec}
\titleformat{\section}
	{\normalfont\Large\bfseries}{\thesection}{1em}{}[{\titlerule[0.8pt]}]
\titleformat{\subsubsection}
	{\normalfont\normalsize\bfseries}{\thesubsubsection}{1em}{}[{\titlerule[0.05pt]}]
\titlespacing{\section}{0pt}{2\parskip}{\parskip}
\titlespacing{\subsection}{0pt}{\parskip}{0pt}
\titlespacing{\subsubsection}{0pt}{\parskip}{0pt}

% Numeración de secciones
\setcounter{tocdepth}{2}
\setcounter{secnumdepth}{2}

% Figuras y descripciones
%\renewcommand{\thefigure}{\Roman{figure}}
\renewcommand{\thesubfigure}{\Alph{subfigure}}
\captionsetup[figure]{labelfont={bf},name={Figura},labelsep=period}
\numberwithin{figure}{chapter}
\numberwithin{equation}{chapter}

% Enumerations
\newcounter{myenumi}
\renewcommand{\themyenumi}{\alph{myenumi})}
\newenvironment{myenumerate}{\setlength{\parindent}{0pt}\setcounter{myenumi}{0}\renewcommand{\item}{\par\refstepcounter{myenumi}\makebox[1.3em][l]{\themyenumi}}}{\par\bigskip\noindent\ignorespacesafterend}

% Own environments
\newenvironment{nota}{\underline{\textbf{NOTA:}} }{}
\newenvironment{caja}{\begin{tcolorbox}[colback = white, sharp corners, boxrule = 1 pt]}{\end{tcolorbox}}
\newtheorem{ejemplo}{Ejemplo}
\newtheorem{ejercicio}{Ejercicio}
\newtheorem*{conclusion}{Conclusión}

% Para una bonita portada
\usepackage{wallpaper}
\usepackage{titling}
\usepackage{fancyhdr}
\pagestyle{fancy}
\setlength{\droptitle}{-10cm} 
\renewcommand{\chaptermark}[1]{%
        \markboth{#1}{}}
\renewcommand{\sectionmark}[1]{%
\markright{\thesection\ #1}}
\fancyhf{}
\fancyhead[LE,RO]{\bfseries\thepage} \fancyhead[LO]{\bfseries\rightmark} \fancyhead[RE]{\bfseries\leftmark} \renewcommand{\headrulewidth}{0.5pt} \renewcommand{\footrulewidth}{0pt} \addtolength{\headheight}{15pt}
 \fancypagestyle{plain}{%
   \fancyhead{} 
   \renewcommand{\headrulewidth}{0pt}
}

% Organización del texto
\newcommand{\formula}[1]{\vspace{13 pt}\noindent \textbf{\underline{#1}}}
\newcommand{\subtext}[1]{_{\text{#1}}}

% Unidades y utilidades varias
\renewcommand{\S}{\operatorname{S}}
\newcommand{\db}{\operatorname{dB}}
\newcommand{\s}{\operatorname{s}}
\newcommand{\A}{\operatorname{A}}
\newcommand{\Pa}{\operatorname{Pa}}
\newcommand{\W}{\operatorname{W}}
\newcommand{\I}{\operatorname{I}}
\newcommand{\C}{\operatorname{C}}
\newcommand{\K}{\operatorname{K}}
\newcommand{\m}{\operatorname{m}}
\newcommand{\rad}{\operatorname{rad}}
\newcommand{\mol}{\operatorname{mol}}
\newcommand{\J}{\operatorname{J}}
\newcommand{\kg}{\operatorname{kg}}
\newcommand{\incremento}{\Delta}
\newcommand{\psus}{\, \ldots \,}
\renewcommand{\sin}{\sen}
\renewcommand{\arcsin}{\arcsen}
\renewcommand{\arctan}{\arctg}
\newcommand{\sen}{\operatorname{\sen}}

% Vectores
\usepackage{esvect}
\renewcommand{\vec}[1]{\vv{{#1}}}
\newcommand{\proy}[2][1]{\text{proy}_{#1}#2}

% Espaciado
\usepackage{enumitem}
\setlist{before={\parskip=3pt}, after=\vspace{\baselineskip}}
\setlength{\parindent}{0pt}

% Estadística
\DeclareMathOperator{\Var}{Var}
\renewcommand{\var}{\sigma ^2}
\DeclareMathOperator{\B}{B}
\DeclareMathOperator{\BN}{BN}
\DeclareMathOperator{\Geo}{Geo}
\DeclareMathOperator{\Poisson}{Poisson}
\DeclareMathOperator{\U}{U}
\DeclareMathOperator{\Exp}{Exp}
\DeclareMathOperator{\N}{N}
\newcommand{\probCond}[2]{P \left( #1 \: \middle\vert\:  #2 \right) }

% Electromagnetismo y Ondas
\newcommand{\errorGrave}{\textbf{FG!!!}}
\newcommand{\mas}{M.A.S.}
\newcommand{\mcu}{M.C.U.}
\newcommand{\ed}{E.D.}
\newcommand{\edmas}{E.D. del M.A.S.}

% Repeticiones
\usepackage{forloop}
\newcommand{\repvec}[3]{
	\foreach \uwu in {1,...,#2}
		{\vec{#1}_{\uwu} ,}
	\, \ldots \, , \vec{#1}_{#3}
}
\newcommand{\rep}[3]{
	\foreach \uwu in {1,...,#2}
		{#1_{\uwu} ,}
	\, \ldots \, , #1_{#3}
}
\newcommand{\repinf}[3]{
	\foreach \uwu in {#2,...,#3}
		{#1_{\uwu} ,}
	\, \ldots 
}

% Señales y Sistemas
\renewcommand{\H}{H}

% Circled number
\newcommand{\circledNumber}[1]{\raisebox{.9pt}{\textcircled{\raisebox{-.9pt}{#1}}}}

% Footnotes
% \renewcommand{\thefootnote}{\fnsymbol{footnote}}

%%%% END OF PREAMBLE %%%%


\title{\Huge Teoría de la Comunicación\\\vspace*{5pt}
\Large Apuntes de clase}
\author{Javier Rodrigo López \thanks{Correo electrónico: \href{mailto:javiolonchelo@gmail.com}{\texttt{javiolonchelo@gmail.com}}}} 
\date{\today}

\setlength{\parskip}{0.5em}

\begin{document}


\setlength{\wpYoffset}{-2 cm}
\ThisCenterWallPaper{0.5}{La_magie_noir-Rene_Magritte.jpg}


\maketitle



\AddToShipoutPictureFG{
\begin{tikzpicture}[overlay,remember picture]
\path (current page.south west) -- (current page.north east)
 node[midway,scale=8,color=lightgray,sloped,opacity=0.05] {Javier Rodrigo López};
\end{tikzpicture}
}




\begin{figure}[t!]
\centering
	\begin{subfigure}[b]{0.65\linewidth}
		\includegraphics[width=\linewidth]{upm_logo.png}
	\end{subfigure}
	\begin{subfigure}[b]{0.25\linewidth}
		\includegraphics[width=\linewidth]{etsist_logo.png}
	\end{subfigure}
\end{figure}


\newpage

\phantomsection

\addcontentsline{toc}{section}{Introducción}
\section*{Introducción}
Imagen de la portada: \textsl{Le magie noire}, por René Magritte.


\newpage

\setlength{\parskip}{0em}
\tableofcontents 
\setlength{\parskip}{0.5em}

\chapter{Modelo de sistema de comunicación}

\chapter{Caracterización de señales}
\section{Representaciones logarítmicas}
\section{Caracterización Temporal}
\section{Caracterización Espectral}
\section{Señales habituales}

\chapter{Ruido térmico}
\section{Caracterización del ruido térmico}
\section{Caracterización del ruido en cuadripolos y dipolos}
\section{Fórmula de Fris}
\section{Modelo de un Analizador de Espectros}

\chapter{Distorsión}
\section{Tipos de distorsión}
\section{Distorsión lineal}
\section{Distorsión no lineal}

\chapter{Modulaciones analógicas}
\section{Concepto de modulación y tipos}
\section{Modulaciones lineales: AM, DBL}
\section{Modulaciones angulares: FM}
\section{Calidad}

\chapter{Conversión A/D y codificación PCM}
\section{\texorpdfstring{Elementos de un sistema de comunicaciones\\ digitales}{Elementos de un sistema de comunicaciones digitales}}
\section{Conversión A/D}
\section{Cuatificación uniforme y no uniforme}
\section{Multiplez por División en el Tiempo (TDM)}

\chapter{Transmisión digital por canales de ancho de banda limitado}
\section{Modelo de Transmisión Digital}
\section{Ancho de banda de señales banda base}
\section{Interferencia entre símbolos (ISI)}
\section{Criterio de Nyquist}
\section{Filtrado en coseno alzado}
\section{Diagrama de ojos}
\section{Códigos de línea}

\chapter{Transmisión digital banda base con ruido}
\section{Representación geométrica de señales}
\section{\texorpdfstring{Implementaciones del receptor: correlador, filtro\\ atrapado}{Implementaciones del receptor: correlador, filtro atrapado}}
\section{Teoría de la Detección (receptor binario óptimo)}
\section{Probabilidad de error en sistemas binarios}
\section[\texorpdfstring{Ejemplos de expresiones de probabilidad de error para varias\\ señalizaciones binarias}{Ejemplos de expresiones de probabilidad de error para varias señalizaciones binarias}]{Ejemplos de expresiones de probabilidad de error para varias señalizaciones binarias}

\addtocontents{toc}{\vfill}
\addtocontents{toc}{\protect\pagebreak}
\chapter{Modulaciones digitales}
\section{Modulaciones lineales. Fórmulas básicas}
\section{ASK}
\section{PSK}
\section{QAM y APK}
\section{FSK}
\section{Comparación entre modulaciones digitales}


% BIBLIOGRAFÍA
%\newpage
%\phantomsection
%\label{sec:bibliografia_final}
%\renewcommand{\refname}{Bibliografía}
%\addcontentsline{toc}{section}{Bibliografía}
%\bibliography{bibliografia} % Nombre del archivo (sin ".bib")
%\bibliographystyle{bababbrv} 

\end{document}
