%%% INICIO DEL PREÁMBULO %%%

\usepackage[utf8]{inputenc}
\usepackage[greek,spanish,es-tabla,es-nodecimaldot,es-noindentfirst]{babel}
\usepackage{fontspec}
\setmonofont{Consolas Bold}
\usepackage{babelbib}
\usepackage{nccmath}
\usepackage[oldvoltagedirection]{circuitikz}
\usepackage{amsthm}
\usepackage{lipsum}
\usepackage{tcolorbox}
\usepackage[thicklines]{cancel}
\usepackage{mathtools}
\usepackage{amssymb}
\usepackage{amsmath}
\usepackage{caption}
\usepackage{subcaption}
\usepackage{color}
\usepackage{verbatim}
\usepackage{enumerate}
\usepackage{geometry}
\geometry{a4paper,left=35mm,right=35mm,top=15mm,bottom=15mm}
\usepackage{isotope}
\usepackage{maybemath}
\usepackage{upgreek}
\usepackage{wasysym}
\usepackage[italic]{hepparticles}
\usepackage{subdepth}
\usepackage[italicdiff]{physics}
\usepackage{braket}
\usepackage{tensor}
\usepackage{chemformula}
\usepackage{tikz} \usetikzlibrary{babel}
\usepackage{siunitx}
\usepackage{url}
\usepackage{listings}
\usepackage{multirow}
\usepackage{multicol}
\usepackage[colorlinks=true]{hyperref}
\hypersetup{
	citecolor = blue,
	linkcolor = blue,
	urlcolor = blue,
	pdfauthor = {Javier Rodrigo López}
}
\usepackage{eso-pic}
\usepackage{siunitx}
\sisetup{
	round-mode      = places,
	round-precision = 2,
}

% tikz
\usepackage{tikz} \usetikzlibrary{fit,babel,shapes,arrows,patterns,positioning,calc,decorations.pathmorphing,decorations.markings}
\tikzstyle{block} = [draw, fill=white, rectangle,
minimum height=3em, minimum width=6em]
\tikzstyle{sum} = [draw, fill=white, circle, node distance=1cm]
\tikzstyle{input} = [coordinate]
\tikzstyle{output} = [coordinate]
\tikzstyle{pinstyle} = [pin edge={to-,thin,black}]
\tikzset{
	block/.style = {draw, fill=white, rectangle, minimum height=3em, minimum width=3em},
	tmp/.style  = {coordinate},
	sum/.style= {draw, fill=white, circle, node distance=1cm},
	input/.style = {coordinate},
	output/.style= {coordinate},
	pinstyle/.style = {pin edge={to-,thin,black}}
}

% Títulos
\usepackage{titlesec}
\titleformat{\section}
{\normalfont\Large\bfseries}{\thesection}{1em}{}[{\titlerule[0.8pt]}]
\titleformat{\subsubsection}
{\normalfont\normalsize\bfseries}{\thesubsubsection}{1em}{}[{\titlerule[0.05pt]}]
\titlespacing{\section}{0pt}{2\parskip}{\parskip}
\titlespacing{\subsection}{0pt}{\parskip}{0pt}
\titlespacing{\subsubsection}{0pt}{\parskip}{0pt}

% Numeración de secciones
\setcounter{tocdepth}{2}
\setcounter{secnumdepth}{2}

% Figuras y descripciones
%\renewcommand{\thefigure}{\Roman{figure}}
\renewcommand{\thesubfigure}{\Alph{subfigure}}
\captionsetup[figure]{labelfont={bf},name={Figura},labelsep=period}
\numberwithin{figure}{chapter}
\numberwithin{equation}{chapter}

% Enumerations
\newcounter{myenumi}
\renewcommand{\themyenumi}{\alph{myenumi})}
\newenvironment{myenumerate}{\setlength{\parindent}{0pt}\setcounter{myenumi}{0}\renewcommand{\item}{\par\refstepcounter{myenumi}\makebox[1.3em][l]{\themyenumi}}}{\par\bigskip\noindent\ignorespacesafterend}

% Own environments
\newenvironment{nota}{\underline{\textbf{NOTA:}} }{}
\newenvironment{caja}{\begin{tcolorbox}[colback = white, sharp corners, boxrule = 1 pt]}{\end{tcolorbox}}
\newtheorem{ejemplo}{Ejemplo}
\newtheorem{ejercicio}{Ejercicio}
\newtheorem*{conclusion}{Conclusión}
\newtheorem{teorema}{Teorema}
\newtheorem{definicion}{Definición}

% Para una bonita portada
\usepackage{wallpaper}
\usepackage{titling}
\usepackage{fancyhdr}
\pagestyle{fancy}
\setlength{\droptitle}{-10cm}
\renewcommand{\chaptermark}[1]{%
	\markboth{#1}{}}
\renewcommand{\sectionmark}[1]{%
	\markright{}}
\fancyhf{}
\fancyhead[LE,RO]{\bfseries\thepage} \fancyhead[LO]{\bfseries\rightmark} \fancyhead[RE]{\bfseries\leftmark} \renewcommand{\headrulewidth}{0pt} \renewcommand{\footrulewidth}{0pt} \addtolength{\headheight}{15pt}
\fancypagestyle{plain}{%
	\fancyhead{}
	\renewcommand{\headrulewidth}{0pt}
}

% Organización del texto
\newcommand{\formula}[1]{\vspace{13 pt}\noindent \textbf{\underline{#1}}}
\newcommand{\subtext}[1]{_{\text{#1}}}

% Unidades y utilidades varias
\renewcommand{\S}{\operatorname{S}}
\newcommand{\dB}{\operatorname{dB}}
\newcommand{\dBW}{\operatorname{dBW}}
\newcommand{\dBm}{\operatorname{dBm}}
\newcommand{\Hz}{\operatorname{Hz}}
\newcommand{\s}{\operatorname{s}}
\newcommand{\A}{\operatorname{A}}
\newcommand{\V}{\operatorname{V}}
\newcommand{\ohm}{\,\Omega}
\newcommand{\Pa}{\operatorname{Pa}}
\newcommand{\W}{\operatorname{W}}
\newcommand{\I}{\operatorname{I}}
\newcommand{\K}{\operatorname{K}}
\newcommand{\m}{\operatorname{m}}
\newcommand{\mm}{\operatorname{mm}}
\newcommand{\rad}{\operatorname{rad}}
\newcommand{\mol}{\operatorname{mol}}
\newcommand{\J}{\operatorname{J}}
\newcommand{\kg}{\operatorname{kg}}
\newcommand{\incremento}{\Delta}
\newcommand{\psus}{\, \ldots \,}
\newcommand{\sen}{\operatorname{\sen}}
\renewcommand{\sin}{\sen}
\renewcommand{\arcsin}{\arcsen}
\renewcommand{\arctan}{\arctg}
\renewcommand{\min}{\operatorname{mín}}

% Vectores
\usepackage[c]{esvect}
\renewcommand{\vec}[1]{\vv{{#1}}}
\newcommand{\proy}[2]{\operatorname{proy}_{\vec{#2}}\vec{#1}}
\newcommand{\antiparallel}{\downharpoonleft \! \upharpoonright}
\newcommand{\parallelvec}{\upharpoonleft \! \upharpoonright}

% Espaciado
\usepackage{enumitem}
\setlist{before={\parskip=3pt}, after=\vspace{\baselineskip}}
\setlength{\parindent}{0pt}
\setlength{\parskip}{0.5em}

% Estadística
\DeclareMathOperator{\Var}{Var}
\DeclareMathOperator{\Cov}{Cov}
\renewcommand{\var}{\sigma ^2}
\DeclareMathOperator{\B}{B}
\DeclareMathOperator{\BN}{BN}
\DeclareMathOperator{\Geo}{Geo}
\DeclareMathOperator{\Poisson}{Poisson}
\DeclareMathOperator{\Exp}{Exp}
\DeclareMathOperator{\N}{N}
\DeclareMathOperator{\Mult}{Mult}
\newcommand{\probCond}[2]{P \left( #1 \: \middle\vert\:  #2 \right) }

% Electromagnetismo y Ondas
\newcommand{\errorGrave}{\textbf{FG!!!}}
\newcommand{\mas}{M.A.S.}
\newcommand{\mcu}{M.C.U.}
\newcommand{\ed}{E.D.}
\newcommand{\edmas}{E.D. del M.A.S.}
\usepackage{esint}

% Señales y Sistemas
\renewcommand{\H}{H}

% Circled number
\newcommand{\circledNumber}[1]{\raisebox{.9pt}{\textcircled{\raisebox{-.9pt}{#1}}}}

% Footnotes
% \renewcommand{\thefootnote}{\fnsymbol{footnote}}

% Repeticiones
\usepackage{forloop}
\newcommand{\repvec}[3]{
	\foreach \uwu in {1,...,#2}
		{\vec{#1}_{\uwu} ,}
	\, \ldots \, , \vec{#1}_{#3}
}
\newcommand{\rep}[3]{
	\foreach \uwu in {1,...,#2}
		{#1_{\uwu} ,}
	\, \ldots \, , #1_{#3}
}
\newcommand{\repinf}[3]{
	\foreach \uwu in {#2,...,#3}
		{#1_{\uwu} ,}
	\, \ldots
}

%%% FIN DEL PREÁMBULO %%%