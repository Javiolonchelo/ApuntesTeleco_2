%%% INICIO DEL PREÁMBULO %%%

\usepackage{amsmath}
\usepackage{amssymb}
\usepackage{amsthm}
\usepackage{babelbib}
\usepackage{braket}
\usepackage{caption}
\usepackage{color}
\usepackage{enumerate}
\usepackage{esint}
\usepackage{eso-pic}
\usepackage{listings}
\usepackage{lscape}
\usepackage{mathtools}
\usepackage{multicol}
\usepackage{multirow}
\usepackage{siunitx}
\usepackage{subcaption}
\usepackage{subdepth}
\usepackage{tcolorbox}
\usepackage{tikz}
\usepackage{titlesec}
\usepackage{titling}
\usepackage{upgreek}
\usepackage{url}
\usepackage{verbatim}
\usepackage{vwcol}
\usepackage{wallpaper}
\usepackage{xfrac}

\usepackage{physics}

\usepackage[c]{esvect}
\usepackage[utf8]{inputenc}
\usepackage[fleqn]{nccmath}
\usepackage[thicklines]{cancel}
\usepackage[margin=2cm]{geometry}
\usepackage[colorlinks=true]{hyperref}
\usepackage[oldvoltagedirection]{circuitikz}
\usepackage[greek,spanish,es-tabla,es-nodecimaldot,es-noindentfirst]{babel}
\sisetup{mode = text}
\hypersetup{
	citecolor = blue,
	linkcolor = blue,
	urlcolor = blue,
	pdfauthor = {Javier Rodrigo López}
}

% tikz
\usetikzlibrary{fit,babel,shapes,arrows,patterns,positioning,calc,decorations.pathmorphing,decorations.markings}
\tikzstyle{block} = [draw, fill=white, rectangle,
minimum height=3em, minimum width=6em]
\tikzstyle{sum} = [draw, fill=white, circle, node distance=1cm]
\tikzstyle{input} = [coordinate]
\tikzstyle{output} = [coordinate]
\tikzstyle{pinstyle} = [pin edge={to-,thin,black}]
\tikzset{
	block/.style = {draw, fill=white, rectangle, minimum height=3em, minimum width=3em},
	tmp/.style  = {coordinate},
	sum/.style= {draw, fill=white, circle, node distance=1cm},
	input/.style = {coordinate},
	output/.style= {coordinate},
	pinstyle/.style = {pin edge={to-,thin,black}}
}

% Títulos
\titleformat{\section}{\normalfont\Large\bfseries}{\thesection}{1em}{}[{\titlerule[0.8pt]}]
% \renewcommand{\thesubsection}{\arabic{chapter}.\arabic{section}.\Alph{subsection}}
\titleformat{\subsubsection}{\normalfont\normalsize\bfseries}{\thesubsubsection}{1em}{}[{\titlerule[0.05pt]}]
\titlespacing{\section}{0pt}{2\parskip}{\parskip}
\titlespacing{\subsection}{0pt}{\parskip}{0pt}
\titlespacing{\subsubsection}{0pt}{\parskip}{0pt}

% Numeración de secciones
\setcounter{tocdepth}{1}
\setcounter{secnumdepth}{2}

% Figuras y descripciones
\captionsetup[figure]{labelfont={bf},name={Figura},labelsep=period}
\numberwithin{figure}{chapter}
\numberwithin{equation}{chapter}
\renewcommand{\thefigure}{\arabic{figure}}
\renewcommand{\thesubfigure}{\Alph{subfigure}}

% Enumerations
\newcounter{myenumi}
\renewcommand{\themyenumi}{\alph{myenumi})}
\newenvironment{myenumerate}{\setlength{\parindent}{0pt}\setcounter{myenumi}{0}\renewcommand{\item}{\par\refstepcounter{myenumi}\makebox[1.3em][l]{\themyenumi}}}{\par\bigskip\noindent\ignorespacesafterend}

% Own environments
\newenvironment{nota}{\underline{\textbf{NOTA:}} }{}
\newenvironment{caja}{\begin{tcolorbox}[colback = white, sharp corners, boxrule = 1 pt]}{\end{tcolorbox}}
\newtheorem*{conclusion}{Conclusión}
\newtheorem{teorema}{Teorema}
\newtheorem{definicion}{Definición}

% Organización del texto
\newcommand{\formula}[1]{\vspace{13 pt}\noindent \textbf{\underline{#1}}}
\newcommand{\subtext}[1]{_{\text{#1}}}

% Unidades y utilidades varias
\renewcommand{\S}{\operatorname{S}}
\newcommand{\dB}{\operatorname{dB}}
\newcommand{\dBW}{\operatorname{dBW}}
\newcommand{\dBm}{\operatorname{dBm}}
\newcommand{\Hz}{\operatorname{Hz}}
\newcommand{\s}{\operatorname{s}}
\newcommand{\A}{\operatorname{A}}
\newcommand{\V}{\operatorname{V}}
\newcommand{\ohm}{\,\Omega}
\newcommand{\Pa}{\operatorname{Pa}}
\newcommand{\W}{\operatorname{W}}
\newcommand{\I}{\operatorname{I}}
\newcommand{\C}{\operatorname{C}}
\newcommand{\K}{\operatorname{K}}
\newcommand{\m}{\operatorname{m}}
\newcommand{\mm}{\operatorname{mm}}
\newcommand{\rad}{\operatorname{rad}}
\newcommand{\mol}{\operatorname{mol}}
\newcommand{\J}{\operatorname{J}}
\newcommand{\kg}{\operatorname{kg}}
\newcommand{\incremento}{\Delta}
\newcommand{\psus}{\, \ldots \,}
\newcommand{\mcm}{\operatorname{mcm}}
\newcommand{\MCD}{\operatorname{MCD}}
\newcommand{\DFT}{\operatorname{DFT}}
\renewcommand{\sin}{\sen}
\renewcommand{\arcsin}{\arcsen}
\renewcommand{\arctan}{\arctg}
\renewcommand{\min}{\operatorname{mín}}
\DeclarePairedDelimiter\evaluat{.}{\rvert}

% Vectores
\renewcommand{\vec}[1]{\vv{{#1}}}
\newcommand{\proy}[2]{\operatorname{proy}_{\vec{#2}}\vec{#1}}
\newcommand{\antiparallel}{\downharpoonleft \! \upharpoonright}
\newcommand{\parallelvec}{\upharpoonleft \! \upharpoonright}

% Espaciado
\usepackage{enumitem}
\setlist{before={\parskip=3pt}, after=\vspace{\baselineskip}}
\setlength{\parindent}{0pt}
\setlength{\parskip}{0.5em}

% Estadística
\DeclareMathOperator{\Var}{Var}
\DeclareMathOperator{\Cov}{Cov}
\renewcommand{\var}{\sigma ^2}
\DeclareMathOperator{\B}{B}
\DeclareMathOperator{\BN}{BN}
\DeclareMathOperator{\Geo}{Geo}
\DeclareMathOperator{\Poisson}{Poisson}
\DeclareMathOperator{\U}{U}
\DeclareMathOperator{\Exp}{Exp}
\DeclareMathOperator{\N}{N}
\DeclareMathOperator{\Mult}{Mult}
\newcommand{\TF}[1]{\mathrm{TF} \left\lbrace \left. #1 \right\rbrace \right.}
\newcommand{\probCond}[2]{P \left( #1 \: \middle\vert\:  #2 \right) }

% Electromagnetismo y Ondas
\newcommand{\errorGrave}{\textbf{FG!!!}}
\newcommand{\mas}{M.A.S.}
\newcommand{\mcu}{M.C.U.}
\newcommand{\ed}{E.D.}
\newcommand{\edmas}{E.D. del M.A.S.}

% Señales y Sistemas
\renewcommand{\H}{H}

% Circled number
\newcommand{\circledNumber}[1]{\raisebox{.9pt}{\textcircled{\raisebox{-.9pt}{#1}}}}

% Footnotes
% \renewcommand{\thefootnote}{\fnsymbol{footnote}}

% Ejemplo
\newcounter{elejemplo}
\newcommand{\ejemplo}[2]{
	\refstepcounter{elejemplo}
	\setlength{\fboxsep}{1em}
	\begin{center}
		\fbox{
			\setlength{\fboxsep}{3pt}
			\begin{minipage}{0.85\linewidth}
			\setlength{\parskip}{0.5em}
			\textbf{Ejemplo \arabic{elejemplo}.} #1
			\begin{center}
				\underline{\textbf{Solución}}
			\end{center}
			#2
		\end{minipage}}
	\end{center}
	\setlength{\fboxsep}{3pt}
}

% Repeticiones
\newcommand{\repvec}[3]{
	\foreach \uwu in {1,...,#2}
		{\vec{#1}_{\uwu} ,}
	\, \ldots \, , \vec{#1}_{#3}
}
\newcommand{\rep}[3]{
	\foreach \uwu in {1,...,#2}
		{#1_{\uwu} ,}
	\, \ldots \, , #1_{#3}
}
\newcommand{\repinf}[3]{
	\foreach \uwu in {#2,...,#3}
		{#1_{\uwu} ,}
	\, \ldots
}

\renewcommand{\arraystretch}{1.5}
\DeclareMathOperator{\circConvUnknown}{\textrm{\scriptsize{\circledNumber{\raisebox{.9pt}{\tiny{N}}}}}}
