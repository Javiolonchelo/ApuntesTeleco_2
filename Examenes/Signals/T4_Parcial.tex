\documentclass{article}
\usepackage[spanish,es-nodecimaldot]{babel}
\usepackage[utf8]{inputenc}
\usepackage{amsmath}
\usepackage{amssymb}
\usepackage{enumitem}
\setlist{itemsep=1em, after=\vspace{\parskip}}
\usepackage{geometry} 
\geometry{a4paper,left=35mm,right=35mm,top=15mm,bottom=15mm}

\setlength{\parskip}{0.5em}

\newcounter{myenumi}
\renewcommand{\themyenumi}{\alph{myenumi})}
\newenvironment{myenumerate}{\setlength{\parindent}{0pt}\setcounter{myenumi}{0}\renewcommand{\item}{\par\refstepcounter{myenumi}\makebox[1.3em][l]{\themyenumi}}}{\par\bigskip\noindent\ignorespacesafterend}

\title{Señales y Sistemas}
\author{Examen del Tema 4}
\date{16 de diciembre de 2020}

\begin{document}
\maketitle

\begin{enumerate}

	\item Siendo $H(\Omega)$ la función de transferencia de un sistema, realiza las siguientes operaciones.
	      \[H(\Omega)=\frac{e^{j\Omega}}{e^{j\Omega}-0.6}\]
	      \begin{enumerate}
		      \item $\displaystyle{h[n-2]}$
		      \item $\displaystyle{(-1)^n\cdot h[n]}$
		      \item $\displaystyle{h[n]\cdot \cos \left( \frac{1}{10} \pi n \right)}$
		      \item $\displaystyle{x[n]\circledast h[n]\, , \ \text{con }X(\Omega) = \delta (\Omega)}$
		      \item $\displaystyle{x[n]\circledast h[n]\, , \ \text{con }x(n) = 2\cos \left( 0.2 \pi n -0.3 \pi \right)}$
		      \item $\displaystyle{x[n]\circledast h[n]\, , \ \text{con }X(\Omega) = 2\cos \left( \Omega \right)}$
	      \end{enumerate}



	\item Realiza la transformada de Laplace de las siguientes señales.

\end{enumerate}

\end{document}