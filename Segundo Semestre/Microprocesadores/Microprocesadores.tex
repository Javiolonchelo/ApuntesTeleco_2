\documentclass[a4paper]{book}

\input{../../Archivos comunes/Preamble.tex} % Se incluye el preámbulo

% Título y portada
\title{\Huge Microprocesadores\\\vspace*{5pt}
\Large Apuntes de clase}
\author{Javier Rodrigo López \thanks{Correo electrónico: \href{mailto:javiolonchelo@gmail.com}{\texttt{javiolonchelo@gmail.com}}}} 
\date{\today}

%%% INICIO DEL DOCUMENTO %%%
\begin{document}

\setlength{\wpYoffset}{-2 cm}
\ThisCenterWallPaper{0.5}{./Imágenes/Bouguereau.jpg}
\maketitle

% Marca de agua
\AddToShipoutPictureFG{
\begin{tikzpicture}[overlay,remember picture]
\path (current page.south west) -- (current page.north east)
 node[midway,scale=8,color=lightgray,sloped,opacity=0.05] {Javier Rodrigo López};
\end{tikzpicture}
}

% Logotipos UPM y ETSIST
\begin{figure}[t!]
\centering
	\begin{subfigure}[b]{0.65\linewidth}
		\includegraphics[width=\linewidth]{../../Archivos comunes/upm_logo.png}
	\end{subfigure}
	\begin{subfigure}[b]{0.25\linewidth}
		\includegraphics[width=\linewidth]{../../Archivos comunes/etsist_logo.png}
	\end{subfigure}
\end{figure}

% Introducción
\newpage
\phantomsection
\addcontentsline{toc}{section}{Introducción}
\section*{Introducción}
Imagen de la portada: \textsl{Dante y Virgilio en el infierno}, por William-Adolphe Bouguereau.
\newpage

% Índice (TOC)
\setlength{\parskip}{0em}
\tableofcontents 
\setlength{\parskip}{0.5em}

%%% INICIO DE LOS APUNTES %%%
\chapter{Memorias semiconductoras}
\section{Bancos de registros}
\section{Memorias semiconductoras}
\subsection{Clasificación}
\subsection{Características}
\subsection{Parámetros}

\section{Mapas de memoria}

\chapter{Microprocesadores}
\section{Concepto de algoritmo}
\section{Sistemas secuenciales con memoria. Definición de microprocesador}
\section{Elemntos internos de un microprocesador}
\section{Arquitectura de tres buses}
\section{Ejemplos de codificación de instrucciones}
\section{Evolución de los microprocesadores}
\section{Modelos de programación y set de instrucciones}
\section{Pila}
\section{Característica de las arquitecturas}
\section{Entorno de programación para sistemas empotrados}

\chapter{Procesador ARM Cortex-M0}
\section{Historia de ARM}
\section{Arquitectura ARM Cortex-M0}
\subsection{Características principales de la arquitectura}
\subsection{Organización de memoria}
\subsection{Modelo de programación}
\subsection{Set de instrucciones}
\subsection{Reset del procesador}
\subsection{Tamaños de datos}
\section{Microcontroladores basados en arquitecturas ARM Cortex-M}
\subsection{NXP LPC1768}
\subsection{STM ST32L432KC}

\chapter{Técnicas de I/O e interrupciones}
\section{Entrada/Salida}
\section{GPIO}
\section{Interrupciones}
\subsection{Polling e interrupciones}
\subsection{Esquemas hardware para la gestión de interrupciones}
\subsection{Esquemas hardware para la gestión de interrupciones}
\subsection{\texorpdfstring{Conceptos de enmascaramiento, vector, prioridad, latencia,\\ anidamiento y excepción}{Conceptos de enmascaramiento, vector, prioridad, latencia, anidamiento y excepción}}
\subsection{Sleep}
\subsection{Particularización para la arquitectura Cortex-M0}
\section{Temporizadores}
\section{PWM}
\section{ADC y DAC}
\section{Sistemas controlados por eventos}
\subsection{Concepto de sistema reactivo y de evento}
\subsection{Máquinas de estados finitos controladas por eventos}
\subsubsection{Eventos y mensajes}
\subsubsection{Estados y variables extendidas, guardas}
\subsubsection{Codificación en C}
\subsubsection{Ejemplo de aplicación completa}
\section{Comunicaciones serie asíncronas}
\subsection{Concepto}
\subsection{Parámetros y variantes}
\subsection{Interfaz físico}
\subsection{UART y transceiver}
\subsection{Programación}

\chapter{Laboratorio}
\section{Lenguaje de ensamble}
\section{Entrada/Salida}
\section{Temporizadores e interrupciones}
\section{Diseño de aplicación de mediana complejidad}
%%% FIN DE LOS APUNTES %%%

%%% BIBLIOGRAFÍA %%%
% Por defecto, se encuentra desactivada. Esto disminuye el tiempo de procesado. Se puede activar cuando se vaya a exportar el PDF definitivo

%\newpage
%\phantomsection
%\label{sec:bibliografia_final}
%\renewcommand{\refname}{Bibliografía}
%\addcontentsline{toc}{section}{Bibliografía}
%\bibliography{bibliografia} % Nombre del archivo (sin ".bib")
%\bibliographystyle{bababbrv} 

\end{document}
