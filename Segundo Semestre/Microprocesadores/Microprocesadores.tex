\documentclass[a4paper]{book}

\usepackage[utf8]{inputenc}
\usepackage[greek,spanish,es-tabla,es-nodecimaldot,es-noindentfirst]{babel}
\usepackage{babelbib}
\usepackage{nccmath}
\usepackage[oldvoltagedirection]{circuitikz}
\usepackage{amsthm}
\usepackage{lipsum}
\usepackage{tcolorbox}
\usepackage[thicklines]{cancel}
\usepackage{mathtools}
\usepackage{amssymb}
\usepackage{amsmath}
\usepackage{caption}
\usepackage{subcaption}   
\usepackage{color}       
\usepackage{verbatim}     
\usepackage{enumerate}
\usepackage{geometry} 
\geometry{a4paper,left=35mm,right=35mm,top=15mm,bottom=15mm}
\usepackage{isotope}
\usepackage{maybemath} 
\usepackage{upgreek}
\usepackage{wasysym} 
\usepackage[italic]{hepparticles}
\usepackage{subdepth}
\usepackage{physics}
\usepackage{braket}
\usepackage{tensor}
\usepackage{chemformula} 
\usepackage{tikz} \usetikzlibrary{babel}
\usepackage{siunitx}
\usepackage{url}
\usepackage{multirow}
\usepackage{multicol}
\usepackage[colorlinks=true]{hyperref}
\hypersetup{
citecolor = blue,
linkcolor = blue,
urlcolor = blue,
pdfauthor = {Javier Rodrigo López}
}
\usepackage{eso-pic}
\usepackage{siunitx}
\sisetup{
round-mode      = places,
round-precision = 2,
}

% Títulos
\usepackage{titlesec}
\titleformat{\section}
	{\normalfont\Large\bfseries}{\thesection}{1em}{}[{\titlerule[0.8pt]}]
\titleformat{\subsubsection}
	{\normalfont\normalsize\bfseries}{\thesubsubsection}{1em}{}[{\titlerule[0.05pt]}]
\titlespacing{\section}{0pt}{2\parskip}{\parskip}
\titlespacing{\subsection}{0pt}{\parskip}{0pt}
\titlespacing{\subsubsection}{0pt}{\parskip}{0pt}

% Numeración de secciones
\setcounter{tocdepth}{2}
\setcounter{secnumdepth}{2}

% Figuras y descripciones
%\renewcommand{\thefigure}{\Roman{figure}}
\renewcommand{\thesubfigure}{\Alph{subfigure}}
\captionsetup[figure]{labelfont={bf},name={Figura},labelsep=period}
\numberwithin{figure}{chapter}
\numberwithin{equation}{chapter}

% Enumerations
\newcounter{myenumi}
\renewcommand{\themyenumi}{\alph{myenumi})}
\newenvironment{myenumerate}{\setlength{\parindent}{0pt}\setcounter{myenumi}{0}\renewcommand{\item}{\par\refstepcounter{myenumi}\makebox[1.3em][l]{\themyenumi}}}{\par\bigskip\noindent\ignorespacesafterend}

% Own environments
\newenvironment{nota}{\underline{\textbf{NOTA:}} }{}
\newenvironment{caja}{\begin{tcolorbox}[colback = white, sharp corners, boxrule = 1 pt]}{\end{tcolorbox}}
\newtheorem{ejemplo}{Ejemplo}
\newtheorem{ejercicio}{Ejercicio}
\newtheorem*{conclusion}{Conclusión}

% Para una bonita portada
\usepackage{wallpaper}
\usepackage{titling}
\usepackage{fancyhdr}
\pagestyle{fancy}
\setlength{\droptitle}{-10cm} 
\renewcommand{\chaptermark}[1]{%
        \markboth{#1}{}}
\renewcommand{\sectionmark}[1]{%
\markright{\thesection\ #1}}
\fancyhf{}
\fancyhead[LE,RO]{\bfseries\thepage} \fancyhead[LO]{\bfseries\rightmark} \fancyhead[RE]{\bfseries\leftmark} \renewcommand{\headrulewidth}{0.5pt} \renewcommand{\footrulewidth}{0pt} \addtolength{\headheight}{15pt}
 \fancypagestyle{plain}{%
   \fancyhead{} 
   \renewcommand{\headrulewidth}{0pt}
}

% Organización del texto
\newcommand{\formula}[1]{\vspace{13 pt}\noindent \textbf{\underline{#1}}}
\newcommand{\subtext}[1]{_{\text{#1}}}

% Unidades y utilidades varias
\renewcommand{\S}{\operatorname{S}}
\newcommand{\db}{\operatorname{dB}}
\newcommand{\s}{\operatorname{s}}
\newcommand{\A}{\operatorname{A}}
\newcommand{\Pa}{\operatorname{Pa}}
\newcommand{\W}{\operatorname{W}}
\newcommand{\I}{\operatorname{I}}
\newcommand{\C}{\operatorname{C}}
\newcommand{\K}{\operatorname{K}}
\newcommand{\m}{\operatorname{m}}
\newcommand{\rad}{\operatorname{rad}}
\newcommand{\mol}{\operatorname{mol}}
\newcommand{\J}{\operatorname{J}}
\newcommand{\kg}{\operatorname{kg}}
\newcommand{\incremento}{\Delta}
\newcommand{\psus}{\, \ldots \,}
\renewcommand{\sin}{\sen}
\renewcommand{\arcsin}{\arcsen}
\renewcommand{\arctan}{\arctg}
\newcommand{\sen}{\operatorname{\sen}}

% Vectores
\usepackage{esvect}
\renewcommand{\vec}[1]{\vv{{#1}}}
\newcommand{\proy}[2][1]{\text{proy}_{#1}#2}

% Espaciado
\usepackage{enumitem}
\setlist{before={\parskip=3pt}, after=\vspace{\baselineskip}}
\setlength{\parindent}{0pt}

% Estadística
\DeclareMathOperator{\Var}{Var}
\renewcommand{\var}{\sigma ^2}
\DeclareMathOperator{\B}{B}
\DeclareMathOperator{\BN}{BN}
\DeclareMathOperator{\Geo}{Geo}
\DeclareMathOperator{\Poisson}{Poisson}
\DeclareMathOperator{\U}{U}
\DeclareMathOperator{\Exp}{Exp}
\DeclareMathOperator{\N}{N}
\newcommand{\probCond}[2]{P \left( #1 \: \middle\vert\:  #2 \right) }

% Electromagnetismo y Ondas
\newcommand{\errorGrave}{\textbf{FG!!!}}
\newcommand{\mas}{M.A.S.}
\newcommand{\mcu}{M.C.U.}
\newcommand{\ed}{E.D.}
\newcommand{\edmas}{E.D. del M.A.S.}

% Repeticiones
\usepackage{forloop}
\newcommand{\repvec}[3]{
	\foreach \uwu in {1,...,#2}
		{\vec{#1}_{\uwu} ,}
	\, \ldots \, , \vec{#1}_{#3}
}
\newcommand{\rep}[3]{
	\foreach \uwu in {1,...,#2}
		{#1_{\uwu} ,}
	\, \ldots \, , #1_{#3}
}
\newcommand{\repinf}[3]{
	\foreach \uwu in {#2,...,#3}
		{#1_{\uwu} ,}
	\, \ldots 
}

% Señales y Sistemas
\renewcommand{\H}{H}

% Circled number
\newcommand{\circledNumber}[1]{\raisebox{.9pt}{\textcircled{\raisebox{-.9pt}{#1}}}}

% Footnotes
% \renewcommand{\thefootnote}{\fnsymbol{footnote}}

%%%% END OF PREAMBLE %%%%


\title{\Huge Microprocesadores\\\vspace*{5pt}
\Large Apuntes de clase}
\author{Javier Rodrigo López \thanks{Correo electrónico: \href{mailto:javiolonchelo@gmail.com}{\texttt{javiolonchelo@gmail.com}}}} 
\date{\today}

\setlength{\parskip}{0.5em}

\begin{document}


\setlength{\wpYoffset}{-2 cm}
\ThisCenterWallPaper{0.5}{Bouguereau.jpg}


\maketitle



\AddToShipoutPictureFG{
\begin{tikzpicture}[overlay,remember picture]
\path (current page.south west) -- (current page.north east)
 node[midway,scale=8,color=lightgray,sloped,opacity=0.05] {Javier Rodrigo López};
\end{tikzpicture}
}




\begin{figure}[t!]
\centering
	\begin{subfigure}[b]{0.65\linewidth}
		\includegraphics[width=\linewidth]{upm_logo.png}
	\end{subfigure}
	\begin{subfigure}[b]{0.25\linewidth}
		\includegraphics[width=\linewidth]{etsist_logo.png}
	\end{subfigure}
\end{figure}


\newpage

\phantomsection

\addcontentsline{toc}{section}{Introducción}
\section*{Introducción}
Imagen de la portada: \textsl{Dante y Virgilio en el infierno}, por William-Adolphe Bouguereau.


\newpage

\setlength{\parskip}{0em}
\tableofcontents 
\setlength{\parskip}{0.5em}

\chapter{Memorias semiconductoras}
\section{Bancos de registros}
\section{Memorias semiconductoras}
\subsection{Clasificación}
\subsection{Características}
\subsection{Parámetros}

\section{Mapas de memoria}

\chapter{Microprocesadores}
\section{Concepto de algoritmo}
\section{Sistemas secuenciales con memoria. Definición de microprocesador}
\section{Elemntos internos de un microprocesador}
\section{Arquitectura de tres buses}
\section{Ejemplos de codificación de instrucciones}
\section{Evolución de los microprocesadores}
\section{Modelos de programación y set de instrucciones}
\section{Pila}
\section{Característica de las arquitecturas}
\section{Entorno de programación para sistemas empotrados}

\chapter{Procesador ARM Cortex-M0}
\section{Historia de ARM}
\section{Arquitectura ARM Cortex-M0}
\subsection{Características principales de la arquitectura}
\subsection{Organización de memoria}
\subsection{Modelo de programación}
\subsection{Set de instrucciones}
\subsection{Reset del procesador}
\subsection{Tamaños de datos}
\section{Microcontroladores basados en arquitecturas ARM Cortex-M}
\subsection{NXP LPC1768}
\subsection{STM ST32L432KC}

\chapter{Técnicas de I/O e interrupciones}
\section{Entrada/Salida}
\section{GPIO}
\section{Interrupciones}
\subsection{Polling e interrupciones}
\subsection{Esquemas hardware para la gestión de interrupciones}
\subsection{Esquemas hardware para la gestión de interrupciones}
\subsection{\texorpdfstring{Conceptos de enmascaramiento, vector, prioridad, latencia,\\ anidamiento y excepción}{Conceptos de enmascaramiento, vector, prioridad, latencia, anidamiento y excepción}}
\subsection{Sleep}
\subsection{Particularización para la arquitectura Cortex-M0}
\section{Temporizadores}
\section{PWM}
\section{ADC y DAC}
\section{Sistemas controlados por eventos}
\subsection{Concepto de sistema reactivo y de evento}
\subsection{Máquinas de estados finitos controladas por eventos}
\subsubsection{Eventos y mensajes}
\subsubsection{Estados y variables extendidas, guardas}
\subsubsection{Codificación en C}
\subsubsection{Ejemplo de aplicación completa}
\section{Comunicaciones serie asíncronas}
\subsection{Concepto}
\subsection{Parámetros y variantes}
\subsection{Interfaz físico}
\subsection{UART y transceiver}
\subsection{Programación}

\chapter{Laboratorio}
\section{Lenguaje de ensamble}
\section{Entrada/Salida}
\section{Temporizadores e interrupciones}
\section{Diseño de aplicación de mediana complejidad}

% BIBLIOGRAFÍA
%\newpage
%\phantomsection
%\label{sec:bibliografia_final}
%\renewcommand{\refname}{Bibliografía}
%\addcontentsline{toc}{section}{Bibliografía}
%\bibliography{bibliografia} % Nombre del archivo (sin ".bib")
%\bibliographystyle{bababbrv} 

\end{document}
