\documentclass[a4paper]{book}

\input{../../Archivos comunes/Preamble.tex} % Se incluye el preámbulo

% Título y portada
\title{\Huge Teoría de la Comunicación\\\vspace*{5pt}
\Large Apuntes de clase}
\author{Javier Rodrigo López \thanks{Correo electrónico: \href{mailto:javiolonchelo@gmail.com}{\texttt{javiolonchelo@gmail.com}}}} 
\date{\today}

%%% INICIO DEL DOCUMENTO %%%
\begin{document}

\setlength{\wpYoffset}{-2 cm}
\ThisCenterWallPaper{0.5}{./Imágenes/La_magie_noir-Rene_Magritte.jpg}
\maketitle

% Marca de agua
\AddToShipoutPictureFG{
\begin{tikzpicture}[overlay,remember picture]
\path (current page.south west) -- (current page.north east)
 node[midway,scale=8,color=lightgray,sloped,opacity=0.05] {Javier Rodrigo López};
\end{tikzpicture}
}

% Logotipos UPM y ETSIST
\begin{figure}[t!]
\centering
	\begin{subfigure}[b]{0.65\linewidth}
		\includegraphics[width=\linewidth]{../../Archivos comunes/upm_logo.png}
	\end{subfigure}
	\begin{subfigure}[b]{0.25\linewidth}
		\includegraphics[width=\linewidth]{../../Archivos comunes/etsist_logo.png}
	\end{subfigure}
\end{figure}

% Introducción
\newpage
\phantomsection
\addcontentsline{toc}{section}{Introducción}
\section*{Introducción}
Imagen de la portada: \textsl{Le magie noire}, por René Magritte.
\newpage

% Índice (TOC)
\setlength{\parskip}{0em}
\tableofcontents 
\setlength{\parskip}{0.5em}

%%% INICIO DE LOS APUNTES %%%
\chapter{Modelo de sistema de comunicación}

\chapter{Caracterización de señales}
\section{Representaciones logarítmicas}
\section{Caracterización Temporal}
\section{Caracterización Espectral}
\section{Señales habituales}

\chapter{Ruido térmico}
\section{Caracterización del ruido térmico}
\section{Caracterización del ruido en cuadripolos y dipolos}
\section{Fórmula de Fris}
\section{Modelo de un Analizador de Espectros}

\chapter{Distorsión}
\section{Tipos de distorsión}
\section{Distorsión lineal}
\section{Distorsión no lineal}

\chapter{Modulaciones analógicas}
\section{Concepto de modulación y tipos}
\section{Modulaciones lineales: AM, DBL}
\section{Modulaciones angulares: FM}
\section{Calidad}

\chapter{Conversión A/D y codificación PCM}
\section{\texorpdfstring{Elementos de un sistema de comunicaciones\\ digitales}{Elementos de un sistema de comunicaciones digitales}}
\section{Conversión A/D}
\section{Cuatificación uniforme y no uniforme}
\section{Multiplez por División en el Tiempo (TDM)}

\chapter{Transmisión digital por canales de ancho de banda limitado}
\section{Modelo de Transmisión Digital}
\section{Ancho de banda de señales banda base}
\section{Interferencia entre símbolos (ISI)}
\section{Criterio de Nyquist}
\section{Filtrado en coseno alzado}
\section{Diagrama de ojos}
\section{Códigos de línea}

\chapter{Transmisión digital banda base con ruido}
\section{Representación geométrica de señales}
\section{\texorpdfstring{Implementaciones del receptor: correlador, filtro\\ atrapado}{Implementaciones del receptor: correlador, filtro atrapado}}
\section{Teoría de la Detección (receptor binario óptimo)}
\section{Probabilidad de error en sistemas binarios}
\section[\texorpdfstring{Ejemplos de expresiones de probabilidad de error para varias\\ señalizaciones binarias}{Ejemplos de expresiones de probabilidad de error para varias señalizaciones binarias}]{Ejemplos de expresiones de probabilidad de error para varias señalizaciones binarias}

\addtocontents{toc}{\vfill}
\addtocontents{toc}{\protect\pagebreak}
\chapter{Modulaciones digitales}
\section{Modulaciones lineales. Fórmulas básicas}
\section{ASK}
\section{PSK}
\section{QAM y APK}
\section{FSK}
\section{Comparación entre modulaciones digitales}
%%% FIN DE LOS APUNTES %%%

%%% BIBLIOGRAFÍA %%%
% Por defecto, se encuentra desactivada. Esto disminuye el tiempo de procesado. Se puede activar cuando se vaya a exportar el PDF definitivo

%\newpage
%\phantomsection
%\label{sec:bibliografia_final}
%\renewcommand{\refname}{Bibliografía}
%\addcontentsline{toc}{section}{Bibliografía}
%\bibliography{bibliografia} % Nombre del archivo (sin ".bib")
%\bibliographystyle{bababbrv}

\end{document}
