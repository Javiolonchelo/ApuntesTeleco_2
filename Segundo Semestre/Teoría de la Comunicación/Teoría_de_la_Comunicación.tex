\documentclass[a4paper]{book}

\input{../../Archivos comunes/Preamble.tex} % Se incluye el preámbulo
\usepackage{etoolbox}
\makeatletter
\pgfcircdeclarebipole{}
    {\ctikzvalof{bipoles/tline/height}}{tline}
    {\ctikzvalof{bipoles/tline/height}}
    {\ctikzvalof{bipoles/tline/width}}
{   
    %% First find distance from startpoint to endpoint
    \pgfpointdiff{\pgfpointanchor{\ctikzvalof{bipole/name}start}{center}}
                 {\pgfpointanchor{\ctikzvalof{bipole/name}end}{center}}
    \pgfmathparse{veclen(\the\pgf@x,\the\pgf@y)}
    %% The coordinate system has been changed so that the origin is at the midpoint and
    %% the line is along the x axis. So shift back by half the length of the line, and 
    %% make the cylinder of width roughly the length of the line, with a 40pt setback
    %% on each side.
    \pgftransformxshift{\pgfmathresult/2-30pt}
    \pgf@circ@res@left=\dimexpr-\pgfmathresult pt+40pt\relax

    %% Here is the original function, copied directly from the source of circuittikz, 
    %% down to next %%
    \pgf@circ@res@step=.2\pgf@circ@res@right % half x axis
    \pgfsetlinewidth{\pgfkeysvalueof{/tikz/circuitikz/bipoles/thickness}\pgfstartlinewidth}
    \pgfpathmoveto{\pgfpoint{\pgf@circ@res@right-\pgf@circ@res@step}{\pgf@circ@res@up}}
    \pgfpathlineto{\pgfpoint{\pgf@circ@res@left+\pgf@circ@res@step}{\pgf@circ@res@up}}
    \pgfpatharc{-90}{90}{-\pgf@circ@res@step and -\pgf@circ@res@up}
    \pgfpathlineto{\pgfpoint{\pgf@circ@res@right-\pgf@circ@res@step}{\pgf@circ@res@down}}
    \pgfsetfillcolor{white}
    \pgfusepath{draw,fill}

    %% I have to fill the figure to block out the original line
    \pgfpathellipse{\pgfpoint{\pgf@circ@res@right-\pgf@circ@res@step}{0}}
                   {\pgfpoint{\pgf@circ@res@step}{0}}
                   {\pgfpoint{0}{-\pgf@circ@res@up}}
    \pgfsetfillcolor{white}
    \pgfusepath{draw, fill}

    %% Redraw part of the line that gets blocked by the cylinder by mistake
    \pgfsetlinewidth{\pgfstartlinewidth}
    \pgfpathmoveto{\pgfpoint{\pgf@circ@res@right-1*\pgf@circ@res@step}{0pt}}
    \pgfpathlineto{\pgfpoint{\pgf@circ@res@right}{0pt}}
    \pgfusepath{draw}
} % Para poder dibujar líneas de transmisión con circutikz

% Título y portada
\title{\Huge Teoría de la Comunicación\\\vspace*{5pt}
\Large Apuntes de clase}
\author{Javier Rodrigo López \thanks{Correo electrónico: \href{mailto:javiolonchelo@gmail.com}{\texttt{javiolonchelo@gmail.com}}}} 
\date{\today}

%%% INICIO DEL DOCUMENTO %%%
\begin{document}

\setlength{\wpYoffset}{-2 cm}
\ThisCenterWallPaper{0.5}{./Imágenes/La_magie_noir-Rene_Magritte.jpg}
\maketitle

% Marca de agua
\AddToShipoutPictureFG{
	\begin{tikzpicture}[overlay,remember picture]
		\path (current page.south west) -- (current page.north east)
		node[midway,scale=8,color=lightgray,sloped,opacity=0.05] {Javier Rodrigo López};
	\end{tikzpicture}
}

% Logotipos UPM y ETSIST
\begin{figure}[t!]
	\centering
	\begin{subfigure}[b]{0.65\linewidth}
		\includegraphics[width=\linewidth]{../../Archivos comunes/upm_logo.png}
	\end{subfigure}
	\begin{subfigure}[b]{0.25\linewidth}
		\includegraphics[width=\linewidth]{../../Archivos comunes/etsist_logo.png}
	\end{subfigure}
\end{figure}

% Introducción
\newpage
\phantomsection
\addcontentsline{toc}{section}{Introducción}
\section*{Introducción}
Imagen de la portada: \textsl{Le magie noire}, por René Magritte.

Esta asignatura es básica para cualquier ingeniería de Telecomunicaciones. Se basa principalmente en las matemáticas explicadas en Señales y Sistemas. Por ello, para las prácticas de laboratorio usaremos MATLAB.

El Bloque 1 representa el 40\%

La evaluación del laboratorio se realizará a partir de los informes de las prácticas $(50\% )$ y del examen $(50\% )$.

Teoría 90\% + LAB 10\%

Repasar presentación en powerpoint para completar introducción.

\newpage

% Índice (TOC)
\setlength{\parskip}{0em}
\tableofcontents
\setlength{\parskip}{0.5em}

%%% INICIO DE LOS APUNTES %%%
\chapter{Modelo de sistema de comunicación}

\section{Definiciones básicas}

La \textbf{ITU} (Unión Internacional de Telecomunicaciones) nos indica la terminología que debemos usar en el ámbito de las telecomunicaciones.

\begin{description}
	\item[Canal de transmisión:] Conjunto de medios necesarios para asegurar la transmisión de señales en un sentido entre dos puntos.
	\item[Señal:] Fenómeno físico en el cual pueden variar una o más características para \textbf{representar información}.
	      \begin{itemize}
		      \item \textbf{Canal de frecuencia:} Parte del espectro de frecuencias que se destina a ser utilizado para la transmisión de señales y que puede determinarse por su frecuencia central y el ancho de banda asociado.
	      \end{itemize}
	\item[Telecomunicación:] Tota transmisión, emisión o recepción de señales que representan signos, escritura, imágenes y sonidos o \textbf{información de cualquier naturaleza} por hilo, ondas electromagnéticas, medios ópticos u otros sistemas electromagnéticos.
	\item[Teoría de la comunicación:] Tiene por objeto encontrar las técnicas más adecuadas que, con los condicionantes económicos, tecnológicos... permiten optimizar el \textbf{consumo de ancho de banda} (BW) y \textbf{potencia} ($P$) para poder transmitir una determinada información con una \textbf{calidad determinada}.
\end{description}

\section{Esquema funcional de un sistema de comunicación}
FALTA AÑADIR IMAGEN

\subsection{Fuentes de información}

Las diferentes fuentes de información pueden clasificarse como:
\begin{description}
	\item[Analógica] La información a transmitir es una señal continua en el tiempo. Cabe mencionar que las señales analógicas pueden digitalizarse. Por ejemplo, una forma de conseguirlo sería mediante cuantificación y codificación PCM (explicado más adelante, falta añadir una referencia cuando lleguemos a esa parte del temario, en el Tema 6).
	\item[Digital] La información consiste en símbolos pertenecientes a un alfabeto finito, que se envían secuencialmente en intervalos discretos de tiempo. Los \textbf{símbolos} son los posibles valores que puede tomar. Por ejemplo, una señal digital binaria tiene dos símbolos.
\end{description}

\subsection{Transmisor}

El transmisor convierte la señal de información (fuente) en señales eléctricas o electromagnéticas (formas de onda) adecuadas para su transmisión a través del medio físico (canal de comunicaciones).

Existen varios tipos de transmisiones:

\begin{itemize}
	\item Transmisión \textbf{banda base} $\longleftrightarrow$ Transmisión paso banda (\textbf{modulación}).
	      \begin{itemize}
		      \item En banda base: Se emite la información en la misma banda que ocupa, como se generó la fuente.
		      \item Con modulación: La banda ocupada por la información se traslada a otra más alta. Esto se hace para:
		            \begin{itemize}
			            \item Adaptar la banda transmitida a los requerimientos del canal.
			            \item Multiplexar señales. Es decir, permitir que varias compartan el mismo canal de comunicaciones. \textbf{FDM} (Multiplex por división en frecuencia).
		            \end{itemize}
	      \end{itemize}
	\item Transmisión \textbf{analógica} $\longleftrightarrow$ Transmisión \textbf{digital}
\end{itemize}

\subsubsection{Modulación}

La señal moduladora modula una señal portadora (sinusoidal en nuestro caso)
\[ S\subtext{moduladora}(t) \]
\[ x_p(t) = A \sen \left( \omega t + \phi \right) \]
\[ \omega _c = 2\pi f_c \]

[Representación del espectro del seno]

\begingroup
\renewcommand{\arraystretch}{1.2}
\begin{center}
	\begin{tabular}{c | c | c}
		Portadora & Analógica        & Digital \\ \hline
		          & AM \footnotemark & ASK     \\
		Senoidal  & FM               & FSK     \\
		          & PM               & PSK     \\ \hline
		          & PAM o PCM        &         \\
		Cuadrada  & PPM              &         \\
		          & PWM              &         \\ \hline
	\end{tabular}
\end{center}
\footnotetext{Modulación en amplitud}
\endgroup

\chapter{Caracterización de señales}

\section{Representaciones logarítmicas}

\subsubsection{Ejercicio 1}

Tenemos un canal de transmisión con un amplificador que ofrece una ganancia $G$ de $32\dB$ y un cable muy largo que afecta aplicando una atenuación $A$ de $20\dB$ a la señal. A la entrada del amplificador (punto 1), introducimos un tono con amplitud de pico $2\V$. Sabemos que la resistencia es de $50 \ohm$.

Rellena la tabla con los valores que se piden para cada parte del canal de transmisión.

\begin{center}
	\begin{circuitikz}
		\draw (0,0) node[label={above:$\circledNumber{1}$}] {}
		to[amp,l_=${G=32\dB}$,o-o] ++(3,0) node[label={above:$\circledNumber{2}$}] {} to[TL,l_=${A=20\dB}$, -o] ++(5,0) node[label={above:$\circledNumber{3}$}] {};
	\end{circuitikz}
\end{center}

\begin{center}
	\underline{\textbf{Solución}}
\end{center}

Las respuestas han sido coloreadas de color verde en la tabla, y a continuación puedes observar la resolución del ejercicio. Existen numerosas formas de llegar al resultado. Esta solución es la que se me ocurrió según lo resolvía.

\begingroup
\renewcommand{\arraystretch}{1.5}
\begin{center}
	\begin{tabular}{|c|c|c|c|}
		\hline
		\textbf{Magnitud} & $\circledNumber{1}$                   & $\circledNumber{2}$                   & $\circledNumber{3}$                   \\ \hline
		$x_p \ [\V ]$     & $2\V$                                 & \textcolor{green!55!black}{$79.62\V$} & \textcolor{green!55!black}{$7.96\V$}  \\ \hline
		$p \ [\W ]$       & \textcolor{green!55!black}{$0.04\W$}  & \textcolor{green!55!black}{$7.94\W$}  & \textcolor{green!55!black}{$0.794\W$} \\ \hline
		$P \ [\dBW ]$     & \textcolor{green!55!black}{$-14\dBW$} & \textcolor{green!55!black}{$18\dBW$}  & \textcolor{green!55!black}{$-2\dBW$}  \\ \hline
		$P \ [\dBm ]$     & \textcolor{green!55!black}{$16\dBm$}  & \textcolor{green!55!black}{$48\dBm$}  & \textcolor{green!55!black}{$18\dBm$}  \\ \hline
	\end{tabular}
\end{center}
\endgroup

Podemos calcular $p_1$ sabiendo que la potencia de un tono es la siguiente:
\[ p_1 = \frac{x_p^2}{2R} = \frac{2^2}{2 \cdot 50} = \boxed{0.04\W }\]

Por lo tanto:
\begin{align*}
	P_1 [\dBW] & = 10 \log \left( 0.04 \right) = \boxed{-14\dBW} \\
	P_1 [\dBm] & = 10 \log \left( 40 \right) = \boxed{16\dBm}
\end{align*}

De aquí, podemos obtener el resto de potencias logarítmicas:
\begin{align*}
	P_2 [\dBW] & = P_1 + G = -14 + 32 = \boxed{18\dBW} \\
	P_2 [\dBm] & = P_1 + G = 16 + 32  = \boxed{48\dBm} \\[10pt]
	P_3 [\dBW] & = P_1 - A = 18 - 20  = \boxed{-2\dBW} \\
	P_3 [\dBm] & = P_1 - A = 48 - 20  = \boxed{18\dBm}
\end{align*}

He decidido obtener los valores restantes de potencia lineal $p$ a partir de las potencias logarítmicas $P$\footnote{Recuerda que $P[\dBW ]=10 \log \left( \frac{p[\W ]}{1\W} \right) \Longrightarrow \boxed{p =10^{\frac{P[\dBW ]}{10}}\W}$}:
\begin{align*}
	p_2 & = 10^{\frac{18}{20}} = \boxed{7.94\W}  \\
	p_3 & = 10^{\frac{-2}{20}} = \boxed{0.794\W}
\end{align*}

Por último, nos queda averiguar los dos restantes valores de pico $x_p$. Haremos uso de la ganancia en tensión $g_v$:
\[ g = \frac{p_2}{p_1} = \frac{V^2_2 / R}{V^2_1 / R} = \left( \frac{V_2}{V_1} \right)^2 = g_v^2 \qquad \Longrightarrow \qquad g_v = \sqrt{g} \]

Sabiendo que la atenuación es simplemente una ganancia negativa, podemos obtener los valores de pico:
\begin{align*}
	x_{p2}  & = x_{p1} \cdot g_v = x_{p1} \cdot \sqrt{g} = 2 \cdot \sqrt{10^{\frac{32}{10}}} = \boxed{79.62\V} \\
	x_{p_3} & = x_{p2} \cdot a = 79.62 \cdot \sqrt{10^{-\frac{20}{10}}} = \boxed{7.962\V}
\end{align*}

\section{Caracterización Temporal}
\section{Caracterización Espectral}

\subsection{Densidad espectral de potencia}

La densidad espectral de potencia $G_x(f)$ mide la potencia de la señal por unidad de ancho de banda $\left( \W / \Hz \right)$.

En un sistema LTI con respuesta en frecuencia $H(f)$, la densidad espectral a la salida se puede calcular como:
\[ \boxed{G_y(f) = G_x(f) \cdot \abs{H(f)}^2} \]

\subsubsection{Ancho de banda}

El \textbf{ancho de banda a 3 dB} se mide entre las frecuencias donde la potencia es la mitad con respecto al máximo.

el \textbf{ancho de banda equivalente} es el ancho de un espectro rectangular ficticio que contendría la misma potencia que la señal original. Es decir, que tiene potencia equivalente. [añadir imagen]

El \textbf{ancho de banda entre nulos} se explicará con más detalle en el Tema 7 [falta referencia]

\subsubsection{Ejercicio 2}

\section{Señales habituales}

\subsection{Señal triangular}

\subsection{Señal cuadrada}

\chapter{Ruido térmico}
\section{Caracterización del ruido térmico}
\section{Caracterización del ruido en cuadripolos y dipolos}
\section{Fórmula de Fris}
\section{Modelo de un Analizador de Espectros}

\chapter{Distorsión}
\section{Tipos de distorsión}
\section{Distorsión lineal}
\section{Distorsión no lineal}

\chapter{Modulaciones analógicas}
\section{Concepto de modulación y tipos}
\section{Modulaciones lineales: AM, DBL}

\subsubsection{Demodulador no coherente de AM}
\subsubsection{Demodulador coherente para modulaciones lineales}

\section{Modulaciones angulares: FM}
\section{Calidad}

\chapter{Conversión A/D y codificación PCM}
\section{\texorpdfstring{Elementos de un sistema de comunicaciones\\ digitales}{Elementos de un sistema de comunicaciones digitales}}
\section{Conversión A/D}
\section{Cuatificación uniforme y no uniforme}
\section{Multiplez por División en el Tiempo (TDM)}

\chapter{Transmisión digital por canales de ancho de banda limitado}
\section{Modelo de Transmisión Digital}
\section{Ancho de banda de señales banda base}
\section{Interferencia entre símbolos (ISI)}
\section{Criterio de Nyquist}
\section{Filtrado en coseno alzado}
\section{Diagrama de ojos}
\section{Códigos de línea}

\chapter{Transmisión digital banda base con ruido}
\section{Representación geométrica de señales}
\section{\texorpdfstring{Implementaciones del receptor: correlador, filtro\\ atrapado}{Implementaciones del receptor: correlador, filtro atrapado}}
\section{Teoría de la Detección (receptor binario óptimo)}
\section{Probabilidad de error en sistemas binarios}
\section[\texorpdfstring{Ejemplos de expresiones de probabilidad de error para varias\\ señalizaciones binarias}{Ejemplos de expresiones de probabilidad de error para varias señalizaciones binarias}]{Ejemplos de expresiones de probabilidad de error para varias señalizaciones binarias}

\addtocontents{toc}{\vfill}
\addtocontents{toc}{\protect\pagebreak}
\chapter{Modulaciones digitales}
\section{Modulaciones lineales. Fórmulas básicas}
\section{ASK}
\section{PSK}
\section{QAM y APK}
\section{FSK}
\section{Comparación entre modulaciones digitales}
%%% FIN DE LOS APUNTES %%%

%%% BIBLIOGRAFÍA %%%
% Por defecto, se encuentra desactivada. Esto disminuye el tiempo de procesado. Se puede activar cuando se vaya a exportar el PDF definitivo

%\newpage
%\phantomsection
%\label{sec:bibliografia_final}
%\renewcommand{\refname}{Bibliografía}
%\addcontentsline{toc}{section}{Bibliografía}
%\bibliography{bibliografia} % Nombre del archivo (sin ".bib")
%\bibliographystyle{bababbrv}

\end{document}
