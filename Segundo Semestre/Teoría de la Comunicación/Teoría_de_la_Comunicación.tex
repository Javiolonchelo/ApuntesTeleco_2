\documentclass[a4paper]{book}

\input{../../Archivos comunes/Preamble.tex} % Se incluye el preámbulo

% Título y portada
\title{\Huge Teoría de la Comunicación\\\vspace*{5pt}
\Large Apuntes de clase}
\author{Javier Rodrigo López \thanks{Correo electrónico: \href{mailto:javiolonchelo@gmail.com}{\texttt{javiolonchelo@gmail.com}}}} 
\date{\today}

%%% INICIO DEL DOCUMENTO %%%
\begin{document}

\setlength{\wpYoffset}{-2 cm}
\ThisCenterWallPaper{0.5}{./Imágenes/La_magie_noir-Rene_Magritte.jpg}
\maketitle

% Marca de agua
\AddToShipoutPictureFG{
\begin{tikzpicture}[overlay,remember picture]
\path (current page.south west) -- (current page.north east)
 node[midway,scale=8,color=lightgray,sloped,opacity=0.05] {Javier Rodrigo López};
\end{tikzpicture}
}

% Logotipos UPM y ETSIST
\begin{figure}[t!]
\centering
	\begin{subfigure}[b]{0.65\linewidth}
		\includegraphics[width=\linewidth]{../../Archivos comunes/upm_logo.png}
	\end{subfigure}
	\begin{subfigure}[b]{0.25\linewidth}
		\includegraphics[width=\linewidth]{../../Archivos comunes/etsist_logo.png}
	\end{subfigure}
\end{figure}

% Introducción
\newpage
\phantomsection
\addcontentsline{toc}{section}{Introducción}
\section*{Introducción}
Imagen de la portada: \textsl{Le magie noire}, por René Magritte.

Esta asignatura es básica para cualquier ingeniería de Telecomunicaciones. Se basa principalmente en las matemáticas explicadas en Señales y Sistemas. Por ello, para las prácticas de laboratorio usaremos MATLAB.

El Bloque 1 representa el 40\%

La evaluación del laboratorio se realizará a partir de los informes de las prácticas $(50\% )$ y del examen $(50\% )$.

Teoría 90\% + LAB 10\%

Repasar presentación en powerpoint para completar introducción.

\newpage

% Índice (TOC)
\setlength{\parskip}{0em}
\tableofcontents 
\setlength{\parskip}{0.5em}

%%% INICIO DE LOS APUNTES %%%
\chapter{Modelo de sistema de comunicación}

\section{Definiciones básicas}

La \textbf{ITU} (Unión Internacional de Telecomunicaciones) nos indica la terminología que debemos usar en el ámbito de las telecomunicaciones.

\begin{description}
	 \item[Canal de transmisión:] Conjunto de medios necesarios para asegurar la transmisión de señales en un sentido entre dos puntos.
	 \item[Señal:] Fenómeno físico en el cual pueden variar una o más características para \textbf{representar información}.
	 \begin{itemize}
		  \item \textbf{Canal de frecuencia:} Parte del espectro de frecuencias que se destina a ser utilizado para la transmisión de señales y que puede determinarse por su frecuencia central y el ancho de banda asociado. 
	 \end{itemize} 
	 \item[Telecomunicación:] Tota transmisión, emisión o recepción de señales que representan signos, escritura, imágenes y sonidos o \textbf{información de cualquier naturaleza} por hilo, ondas electromagnéticas, medios ópticos u otros sistemas electromagnéticos.
	 \item[Teoría de la comunicación:] Tiene por objeto encontrar las técnicas más adecuadas que, con los condicionantes económicos, tecnológicos... permiten optimizar el \textbf{consumo de ancho de banda} (BW) y \textbf{potencia} ($P$) para poder transmitir una determinada información con una \textbf{calidad determinada}.
\end{description}

\section{Esquema funcional de un sistema de comunicación}
FALTA AÑADIR IMAGEN 

\subsection{Fuentes de información}

Las diferentes fuentes de información pueden clasificarse como:
\begin{description}
	 \item[Analógica] La información a transmitir es una señal continua en el tiempo. Cabe mencionar que las señales analógicas pueden digitalizarse. Por ejemplo, una forma de conseguirlo sería mediante cuantificación y codificación PCM (explicado más adelante, falta añadir una referencia cuando lleguemos a esa parte del temario, en el Tema 6).
	 \item[Digital] La información consiste en símbolos pertenecientes a un alfabeto finito, que se envían secuencialmente en intervalos discretos de tiempo. Los \textbf{símbolos} son los posibles valores que puede tomar. Por ejemplo, una señal digital binaria tiene dos símbolos.
\end{description}

\subsection{Transmisor}

El transmisor convierte la señal de información (fuente) en señales eléctricas o electromagnéticas (formas de onda) adecuadas para su transmisión a través del medio físico (canal de comunicaciones).

Existen varios tipos de transmisiones:

\begin{itemize}
	 \item Transmisión \textbf{banda base} $\longleftrightarrow$ Transmisión paso banda (\textbf{modulación}).
	 \begin{itemize}
		  \item En banda base: Se emite la información en la misma banda que ocupa, como se generó la fuente.
		  \item Con modulación: La banda ocupada por la información se traslada a otra más alta. Esto se hace para:
		  \begin{itemize}
			   \item Adaptar la banda transmitida a los requerimientos del canal.
			   \item Multiplexar señales. Es decir, permitir que varias compartan el mismo canal de comunicaciones. \textbf{FDM} (Multiplex por división en frecuencia).
		  \end{itemize}
	 \end{itemize}
	 \item Transmisión \textbf{analógica} $\longleftrightarrow$ Transmisión \textbf{digital}
\end{itemize}

\subsubsection{Modulación}

La señal moduladora modula una señal portadora (sinusoidal en nuestro caso)
\[ S\subtext{moduladora}(t) \]
\[ x_p(t) = A \sen \left( \omega t + \phi \right) \]
\[ \omega _c = 2\pi f_c \]

[Representación del espectro del seno]

\begingroup
\renewcommand{\arraystretch}{1.2}
\begin{center}
	\begin{tabular}{c | c | c}
		Portadora & Analógica & Digital\\ \hline
		& AM \footnotemark & ASK\\ 
		Senoidal & FM & FSK \\ 
		 & PM & PSK \\ \hline
		 & PAM o PCM & \\ 
		Cuadrada & PPM &  \\ 
		 & PWM &  \\ \hline
	\end{tabular}
\end{center}
\footnotetext{Modulación en amplitud}
\endgroup

\chapter{Caracterización de señales}
\section{Representaciones logarítmicas}
\section{Caracterización Temporal}
\section{Caracterización Espectral}
\section{Señales habituales}

\chapter{Ruido térmico}
\section{Caracterización del ruido térmico}
\section{Caracterización del ruido en cuadripolos y dipolos}
\section{Fórmula de Fris}
\section{Modelo de un Analizador de Espectros}

\chapter{Distorsión}
\section{Tipos de distorsión}
\section{Distorsión lineal}
\section{Distorsión no lineal}

\chapter{Modulaciones analógicas}
\section{Concepto de modulación y tipos}
\section{Modulaciones lineales: AM, DBL}
\section{Modulaciones angulares: FM}
\section{Calidad}

\chapter{Conversión A/D y codificación PCM}
\section{\texorpdfstring{Elementos de un sistema de comunicaciones\\ digitales}{Elementos de un sistema de comunicaciones digitales}}
\section{Conversión A/D}
\section{Cuatificación uniforme y no uniforme}
\section{Multiplez por División en el Tiempo (TDM)}

\chapter{Transmisión digital por canales de ancho de banda limitado}
\section{Modelo de Transmisión Digital}
\section{Ancho de banda de señales banda base}
\section{Interferencia entre símbolos (ISI)}
\section{Criterio de Nyquist}
\section{Filtrado en coseno alzado}
\section{Diagrama de ojos}
\section{Códigos de línea}

\chapter{Transmisión digital banda base con ruido}
\section{Representación geométrica de señales}
\section{\texorpdfstring{Implementaciones del receptor: correlador, filtro\\ atrapado}{Implementaciones del receptor: correlador, filtro atrapado}}
\section{Teoría de la Detección (receptor binario óptimo)}
\section{Probabilidad de error en sistemas binarios}
\section[\texorpdfstring{Ejemplos de expresiones de probabilidad de error para varias\\ señalizaciones binarias}{Ejemplos de expresiones de probabilidad de error para varias señalizaciones binarias}]{Ejemplos de expresiones de probabilidad de error para varias señalizaciones binarias}

\addtocontents{toc}{\vfill}
\addtocontents{toc}{\protect\pagebreak}
\chapter{Modulaciones digitales}
\section{Modulaciones lineales. Fórmulas básicas}
\section{ASK}
\section{PSK}
\section{QAM y APK}
\section{FSK}
\section{Comparación entre modulaciones digitales}
%%% FIN DE LOS APUNTES %%%

%%% BIBLIOGRAFÍA %%%
% Por defecto, se encuentra desactivada. Esto disminuye el tiempo de procesado. Se puede activar cuando se vaya a exportar el PDF definitivo

%\newpage
%\phantomsection
%\label{sec:bibliografia_final}
%\renewcommand{\refname}{Bibliografía}
%\addcontentsline{toc}{section}{Bibliografía}
%\bibliography{bibliografia} % Nombre del archivo (sin ".bib")
%\bibliographystyle{bababbrv}

\end{document}
