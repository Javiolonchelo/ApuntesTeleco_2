\documentclass[a4paper]{book}

\input{../../Archivos comunes/Preamble.tex} % Se incluye el preámbulo

% Título y portada
\title{\Huge Fundamentos de Sonido e Imagen\\\vspace*{5pt}
\Large Apuntes de clase}
\author{Javier Rodrigo López \thanks{Correo electrónico: \href{mailto:javiolonchelo@gmail.com}{\texttt{javiolonchelo@gmail.com}}}} 
\date{\today}

%%% INICIO DEL DOCUMENTO %%%
\begin{document}

\setlength{\wpYoffset}{1cm}
\ThisCenterWallPaper{0.7}{./Imágenes/Sorolla.jpg}
\maketitle

% Marca de agua
\AddToShipoutPictureFG{
\begin{tikzpicture}[overlay,remember picture]
\path (current page.south west) -- (current page.north east)
 node[midway,scale=8,color=lightgray,sloped,opacity=0.05] {Javier Rodrigo López};
\end{tikzpicture}
}

% Logotipos UPM y ETSIST
\begin{figure}[t!]
\centering
	\begin{subfigure}[b]{0.65\linewidth}
		\includegraphics[width=\linewidth]{../../Archivos comunes/upm_logo.png}
	\end{subfigure}
	\begin{subfigure}[b]{0.25\linewidth}
		\includegraphics[width=\linewidth]{../../Archivos comunes/etsist_logo.png}
	\end{subfigure}
\end{figure}

% Introducción
\newpage
\phantomsection
\addcontentsline{toc}{section}{Introducción}
\section*{Introducción}
Imagen de la portada: \textsl{Corriendo por la playa. Valencia}, de Joaquín Sorolla.
\newpage

% Índice (TOC)
\setlength{\parskip}{0em}
\tableofcontents 
\setlength{\parskip}{0.5em}

%%% INICIO DE LOS APUNTES %%%
\chapter{Señales, sistemas y medidas acústicas. Revisión de conceptos}

\chapter{Señales acústicas}
\section{Valor RMS y nivel de una señal}
\section{Serie de Fourier y Transformada de Fourier}
\section{Densidad espectral de potencia}
\section{Nivel espectral y nivel en banda}
\section{Ruido blanco y ruido rosa}
\section{Sistemas y medidas acústicas}
\section{Sistema lineal. Función de transferencia. Respuesta al impulso}
\section{Métodos de análisis de sistemas}
\section{Analogías electro-mecánico-acústicas}

\chapter{Audición y voz}
\section{Fisiología y funcionamiento del sistema auditivo humano}
\section{Características de la respuesta auditiva}
\section{No linealidad del sistema auditivo}
\section{Efecto de enmascaramiento temporal y frecuencial}
\section{Audición binaural}
\section{Mecanismo de generación de la voz}
\section{Características acústicas de voz}
\section{Análisis de la señal de voz}

\chapter{Ondas planas y esféricas}
\section{Ecuación de onda plana. Velocidad de propagación}
\section{Velocidad vibratoria e impedancia de una onda plana}
\section{Presión e intensidad acústicas}
\section{Ecuación de una onda esférica}
\section[Velocidad vibratoria e impedancia de una onda esférica]{Velocidad vibratoria e impedancia de una onda\\ esférica}
\section{Campo acústico originado por una fuente. Divergencia esférica}
\section{Potencia radiada por una fuente}

\chapter{Ondas estacionarias}
\section{Reflexión de una onda plana}
\section{Impedancia de una línea de transmisión acústica}
\section{Intensidad acústica de una onda estacionaria}
\section{Transmisión acústica a través de varios medios}

\chapter{Formación y captación de imágenes}
\section{Introducción}
\section{Óptica geométrica}
\section{Fotografía}
\section{Fotometría}

\chapter{El sistema visual humano. Colorimetría}
\section{Introducción a la visión}
\section{Estructura y óptica del ojo humano}
\section{La retina, nuestro sensor}
\section{Percepción: Implicaciones en los sistemas de imagen}

\chapter{Señales utilizadas para la representación de imágenes}
\section{Modelos cromáticos para el almacenamiento cuantificado de los colores}
\section{Señales de luminancia y de crominancia}
\section{Importancia concedida por el ojo a las señales de luminancia y crominancia}
\section{Cartas de barras para los estudios cromáticos de imágenes fijas y de vídeo}
\section{Relación de aspecto y exploraciones progresivas y entrelazada}
\section[Resolución horizontal y vertical de las imágenes (SD, HD, UHD)]{Resolución horizontal y vertical de las imágenes\\ (SD, HD, UHD)}
\section{Señales normalmente utilizadas para la transmisión de señales de vídeo}
\section{Intervalos de vídeo e intervalos de sincronismo}

\appendix

\chapter{Prácticas}
\section[Introducción. Técnicas de medidas acústicas. Técnicas de análisis de sistemas mecánicos y acústicos]{Introducción. Técnicas de medidas acústicas.\\ Técnicas de análisis de sistemas mecánicos y acústicos}
\section{Osciladores mecánicos y acústicos}
\section{Ondas acústicas esféricas. Potencia radiada por una fuente}
\section{Ondas acústicas estacionarias. Impedancia acústica. Impedancia de radiación de un tubo}
\section{Imagen digital}
\section{Relación de aspecto y adaptaciones}
\section{Brillo y contraste}
\section{Color. Saturación y tinte}
%%% FIN DE LOS APUNTES %%%

%%% BIBLIOGRAFÍA %%%
% Por defecto, se encuentra desactivada. Esto disminuye el tiempo de procesado. Se puede activar cuando se vaya a exportar el PDF definitivo

%\newpage
%\phantomsection
%\label{sec:bibliografia_final}
%\renewcommand{\refname}{Bibliografía}
%\addcontentsline{toc}{section}{Bibliografía}
%\bibliography{bibliografia} % Nombre del archivo (sin ".bib")
%\bibliographystyle{bababbrv} 
\end{document}
