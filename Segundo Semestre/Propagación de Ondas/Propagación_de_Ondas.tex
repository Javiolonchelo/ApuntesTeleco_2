\documentclass[a4paper]{book}

\input{../../Archivos comunes/Preamble.tex} % Se incluye el preámbulo

% Título y portada
\title{\Huge Propagación de Ondas\\\vspace*{5pt}
\Large Apuntes de clase}
\author{Javier Rodrigo López \thanks{Correo electrónico: \href{mailto:javiolonchelo@gmail.com}{\texttt{javiolonchelo@gmail.com}}}} 
\date{\today}

%%% INICIO DEL DOCUMENTO %%%
\begin{document}

\setlength{\wpYoffset}{-2 cm}
\ThisCenterWallPaper{0.5}{./Imágenes/Goya.jpg}
\maketitle

% Marca de agua
\AddToShipoutPictureFG{
\begin{tikzpicture}[overlay,remember picture]
\path (current page.south west) -- (current page.north east)
 node[midway,scale=8,color=lightgray,sloped,opacity=0.05] {Javier Rodrigo López};
\end{tikzpicture}
}

% Logotipos UPM y ETSIST
\begin{figure}[t!]
\centering
	\begin{subfigure}[b]{0.65\linewidth}
		\includegraphics[width=\linewidth]{../../Archivos comunes/upm_logo.png}
	\end{subfigure}
	\begin{subfigure}[b]{0.25\linewidth}
		\includegraphics[width=\linewidth]{../../Archivos comunes/etsist_logo.png}
	\end{subfigure}
\end{figure}

% Introducción
\newpage
\phantomsection
\addcontentsline{toc}{section}{Introducción}
\section*{Introducción}
Imagen de la portada: \textsl{Saturno devorando a su hijo}, de Franciso de Goya.
\newpage

% Índice (TOC)
\setlength{\parskip}{0em}
\tableofcontents 
\setlength{\parskip}{0.5em}

%%% INICIO DE LOS APUNTES %%%
\chapter{Operadores vectoriales}
\section{Gradiente de un campo escalar}
\section{Divergencia y rotacional de un campo vectorial}
\section{Teoremade Helmholtz}

\chapter{Ondas acústicas planas}
\section{Notación compleja}
\section{Ecuación de onda. Solución armónica}
\section{Densidad de energía. Intensidad acústica}

\chapter{Ondas acústicas esféricas}
\section{Ecuación de onda esférica}
\section{Solución armónica. Variables acústicas de una onda esférica}
\section{Intensidad de una onda esférica}

\chapter{Reflexión y refracción de una onda plana}
\section{Reflexión y transmisión de una onda plana}
\section{Ondas acústicas estacionarias}
\section{Impedancia de una onda estacionaria}

\chapter{Ecuaciones de Maxwell. Ecuación de onda. Energía}
\section{Ecuaciones de Maxwell en forma diferencial}
\section{Potenciales eléctrico y magnético}
\section{Ecuación de onda para los campos y para los potenciales}
\section{Energía del campo electromagnético. Teorema de Poynting}
\section{Aplicación: Radiación de un dipolo oscilante}

\chapter{Propagación de ondas electromagnéticas en medios conductores}
\section{Solución para ondas planas}
\section{Impedancia e índice de refracción del medio}
\section{Propagación de la energía}
\section{Polarización}

\chapter{Propagación de ondas electromagnéticas en medios conductores}
\section{Densidad de carga libre en el conductor. Carácter transversal}
\section{Solución para ondas planas. Magnitudes complejas}
\section{Balance energético}

\chapter{Reflexión y refracción de ondas electromagnéticas}
\section[Reflexión y refracción en la frontera dieléctrico-dieléctrico]{Reflexión y refracción en la frontera\\ dieléctrico-dieléctrico}
\section{Ecuaciones de Fresnel}
\section{Coeficientes de reflexión y refracción}
\section[Reflexión y refracción en la frontera dieléctrico-conductor]{Reflexión y refracción en la frontera\\ dieléctrico-conductor}

\chapter{Ondas guiadas}
\section{\texorpdfstring{Ondas estacionarias producidas por reflexión en la\\ frontera dieléctrico-conductor. Ondas TE y TM}{Ondas estacionarias producidas por reflexión en la frontera dieléctrico-conductor. Ondas TE y TM}}
\section{Guía de onda formada por dos planos conductores paralelos}
\section{Balance de energía}
\section{Guía de onda rectangular}
%%% FIN DE LOS APUNTES %%%

%%% BIBLIOGRAFÍA %%%
% Por defecto, se encuentra desactivada. Esto disminuye el tiempo de procesado. Se puede activar cuando se vaya a exportar el PDF definitivo

%\newpage
%\phantomsection
%\label{sec:bibliografia_final}
%\renewcommand{\refname}{Bibliografía}
%\addcontentsline{toc}{section}{Bibliografía}
%\bibliography{bibliografia} % Nombre del archivo (sin ".bib")
%\bibliographystyle{bababbrv}

\end{document}
