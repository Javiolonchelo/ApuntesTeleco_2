\documentclass[a4paper]{book}

\input{../../Archivos comunes/Preamble.tex} % Se incluye el preámbulo

% Título y portada
\title{\Huge Procesado Digital de la Señal\\\vspace*{5pt}
\Large Apuntes de clase}
\author{Javier Rodrigo López \thanks{Correo electrónico: \href{mailto:javiolonchelo@gmail.com}{\texttt{javiolonchelo@gmail.com}}}} 
\date{\today}

%%% INICIO DEL DOCUMENTO %%%
\begin{document}

\setlength{\wpYoffset}{-0.5cm}
\ThisCenterWallPaper{0.7}{./Imágenes/Velazquez.jpg}
\maketitle

\titleformat{\chapter}[block]
{\normalfont\Huge\bfseries}{TEMA \thechapter.}{1em}{}

% Marca de agua
\AddToShipoutPictureFG{
	\begin{tikzpicture}[overlay,remember picture]
		\path (current page.south west) -- (current page.north east)
		node[midway,scale=8,color=lightgray,sloped,opacity=0.05] {Javier Rodrigo López};
	\end{tikzpicture}
}

% Logotipos UPM y ETSIST
\begin{figure}[t!]
	\centering
	\begin{subfigure}[b]{0.65\linewidth}
		\includegraphics[width=\linewidth]{../../Archivos comunes/upm_logo.png}
	\end{subfigure}
	\begin{subfigure}[b]{0.25\linewidth}
		\includegraphics[width=\linewidth]{../../Archivos comunes/etsist_logo.png}
	\end{subfigure}
\end{figure}

% Introducción
\newpage
\phantomsection
\addcontentsline{toc}{section}{Introducción}
\section*{Introducción}
Imagen de la portada: \textsl{La fragua de Vulcano}, por Diego Velázquez.

Las clases en las que se basan estos apuntes fueron impartidas por:
\begin{itemize}
	\item  \href{mailto:cesar.diazm@upm.es}{César Díaz Martín}
	\item \href{mailto:eduardo.latorre.iglesias@upm.es}{Eduardo Latorre Iglesias}
\end{itemize}

\newpage

% Índice (TOC)
\setlength{\parskip}{0em}
\tableofcontents
\setlength{\parskip}{0.5em}

%%% INICIO DE LOS APUNTES %%%

\chapter{Transformada de Fourier y Muestreo}

\section{Introducción a la Transformada de Fourier}

La transformada de Fourier nos servirá para representar señales como combinación lineal de unos ciertos componentes o señales básicas.

La aplicación fundamental del análisis de Fourier es el estudio de las señales en el dominio de la frecuencia (\textbf{dominio espectral}).

El análisis de Fourier se puede realizar para:
\begin{itemize}
	\item Señales continuas:
	      \begin{itemize}
		      \item Periódicas: Serie de Fourier de tiempo continuo
		      \item No periódicas: Transformada de Fourier en tiempo discreto
	      \end{itemize}
	\item Señales discretas
\end{itemize}

\section{Series de Fourier de señales periódicas}
Sea $x(t)$ una señal periódica, de periodo $T_o$. Esta señal puede expresarse como una combinación lineal de exponenciales complejas (serie de Fourier).

La \textbf{ecuación de síntesis} es la siguiente:
\begin{equation}
	x(t) = \sum_{k=-\infty}^{\infty}a_k e^{jk\omega _o t}
\end{equation}

Los pesos $a_k$ se denominan \textbf{coeficientes espectrales} y determinan qué cantidad de energía reside en cada frecuencia. Se calculan mediante la \textbf{ecuación de análisis}:
\[  \]

Notas.
\begin{itemize}
	\item Coeficientes hermíticos
	\item baia la cabaia
\end{itemize}

Añadir ejemplos para el seno y el coseno (se hace mediante la fórmula de Euler).

\subsection{Fenómeno de Gibbs}

En una señal que necesite infinitos coeficientes de Fourier para ser representada de forma exacta, al coger solo un conjunto de los coeficientes se produce un "rizado" en ciertos puntos de la señal, con amplitud independiente del número de coeficientes escogidos. Esto es el fenómeno de Gibbs

\section{Transformada de Fourier para señales no periódicas}

Si tenemos una señal no periódica. Si intentamos hacerla periódica repitiéndola cada $T$ y luego hacemos que este tiempo $T$ tienda a infinito, tenemos la transformada de Fourier.

AÑADIR FÓRMULAS
Ecuación de síntesis (transformada de Fourier inversa)
Ecuación de análisis

\subsection{Propiedades de la transformada de Fourier}
HAY QUE EXPLICARLAS

Propiedad de convolución: Convolución en el tiempo es multiplicación en frecuencia.

Propiedad de modulación: Multiplicación en el tiempo es convolución en frecuencia.

Añadir ejemplo

\section{Transformada de Fourier de señales periódicas}

La transformada de Fourier de una señal periódica es una combinación lineal de deltas.

EJEMPLO IMPORTANTE (en este ejemplo se basa el muestreo)
Se calcula $a_k$
\subsection{Algunos pares transformados}

La transformada de un pulso siempre es una sinc.
La transformada inversa de una sinc siempre es un pulso.

\chapter{Procesado digital de señales analógicas}

\section{Tipos de señal} \vspace{\parskip}

\subsection{Señales continuas}

En una señal periódica, podemos obtener el \textbf{periodo} como el mínimo común múltiplo de los periodos de sus componentes. Un ejemplo:

\[ x(t) = 3 \sen \left( 5\pi t + \frac{\pi}{4} \right) + \cos \left( 10\pi t \right)\, ;\qquad \omega_1 = 5\pi \, , \quad \omega_2 = 10\pi \]
\[ T = \mcm \left( \frac{2\pi}{\omega_1} \, , \frac{2\pi}{\omega_2} \right) = \mcm \left( \frac{2\pi}{5\pi} \, , \frac{2\pi}{10\pi} \right) = \mcm \left( \frac{2}{5} \, , \frac{1}{5} \right) = \frac{1}{5} \mcm \left( 2 \, , 1 \right) = \boxed{\frac{2}{5} \, \unit{\s}}\]

\subsection{Señales discretas}

El \textbf{periodo} de una señal discreta se obtiene de forma similar a las señales continuas. Sin embargo, cuando el resultado es una fracción el periodo se calcula como el numerador de dicha fracción.

Siguiendo con el ejemplo anterior, a partir de la última fracción obtenida:
\[ \frac{N}{m} = \frac{2}{5} \quad \Longrightarrow \quad \boxed{N = 2} \]

\section{Filtros}

Un \textbf{filtro} es cualquier sistema que modifica la forma de la señal, tanto en su amplitud como en su fase, del modo deseado. Los catalogamos como \textbf{FIR}\footnote{FIR $\equiv$ \textit{Finite Impulse Response.}} o \textbf{IIR}\footnote{IIR $\equiv$ \textit{Infinite Impulse Response.}}, en función de su respuesta al impulso.

Hablaremos de ellos con profundidad en el Tema \ref{temaDeFiltros}.

\section{Estructura de un sistema PDS}

A continuación, se encuentra representado el esquema general de un sistema PDS:

\begin{figure}[ht]
	\centering
	\begin{tikzpicture}[auto,>=latex']
		\node [input, name=input] (input) {};
		\node [block, below of=input, node distance=0cm] (A) {$
		\begin{matrix}
			\text{Filtro}\\
			\text{anti-aliasing}
		\end{matrix}$};
		\node [block, right of=A, node distance=2.5cm] (B) {ADC};
		\node [block, right of=B, node distance=2.5cm] (C) {$
		\begin{matrix}
			\text{Proceso}\\
			\text{digital}
		\end{matrix}$};
		\node [block, right of=C, node distance=2.5cm] (D) {DAC};
		\node [block, right of=D, node distance=2.5cm] (E) {$
		\begin{matrix}
			\text{Filtro de}\\
			\text{salida}
		\end{matrix}$};

		\draw [-angle 90] (-2,0) --  node{$x(t)$} (A);
		\draw [-angle 90] (A) --  node{$x'(t)$} (B);
		\draw [-angle 90] (B) --  node{$x[n]$} (C);
		\draw [-angle 90] (C) --  node{$y[n]$} (D);
		\draw [-angle 90] (D) --  node{$y'(t)$} (E);
		\draw [-angle 90] (E) --  node{$y(t)$} ++(2,0);
	\end{tikzpicture}
	\caption{Esquema de un sistema PDS} \label{fig:esquema_PDS}
\end{figure}

\newpage

\section{Muestreo}

El muestreo es la recogida de muestras temporalmente equiespaciadas de una señal analógica para conformar una señal discreta.

\subsection{Teorema de Nyquist}

Para realizar el muestreo de una señal sin que se produzca aliasing\footnote{El \textit{aliasing} es el solapamiento espectral producido por el submuestreo.} debemos asegurar que se cumple el \textbf{Teorema de Nyquist}. Este afirma que la frecuencia de muestreo debe ser, al menos, el doble de la frecuencia máxima de la señal a muestrear\footnote{El subíndice ``s'' viene de \textit{sampling} (en inglés, muestreo). El subíndice ``m'' viene de \textit{max}.}:
\[ w_s \geq 2w_m \]

Un ejemplo curioso es el caso de los CDs de audio. Como los humanos somos capaces de escuchar hasta \SI{20}{\kHz} (aproximadamente), los CDs tienen una frecuencia de muestreo un poquito mayor al doble de esta: \SI{44.1}{\kHz}.

\subsection{Evitar solapamiento}

Sabiendo que si no se cumple el teorema de Nyquis (debido al submuestreo) podemos llegar a experimentar solapamiento espectral, tenemos dos alternativas para enfrentarnos a este problema.
\begin{itemize}
	\item \textbf{Aumentar la frecuencia de muestreo}. Es la opción más directa y sencilla. Sin embargo, muchas veces los sistemas que usemos serán limitados. Esta no es una opción que siempre podamos usar.
	\item Utilizar un \textbf{filtro anti-solapamiento}. El uso de un filtro paso bajo con frecuencia de corte $\omega_c = \omega_s/2$ asegura el procesado de la señal limpia. En la práctica, la pérdida de señal es menos perjudicial que los efectos del aliasing.
\end{itemize}

\section{Sistemas equivalentes. C/D y D/C)}

Para trabajar con sistemas continuos y discretos, se deben establecer unos criterios de equivalencia. En resumidas cuentas, los sistemas equivalentes se basan en la siguiente característica:

\[ H_D \left( e^{j\omega } \right) = \biggl. H_C \left( j\omega \right)  \biggr\vert _{\omega = \frac{\Omega}{T_s}} \]

\begin{itemize}
	\item La propiedad de \textbf{invarianza temporal} surge de la relación entre el tiempo continuo y el tiempo discreto: $t = nT_s $
\end{itemize}


\ejemplo
{
	Obtenga el sistema discreto que permite implementar el siguiente sistema LTI continuo:
	\[ H_C \left( j\omega \right) = \left\lbrace
		\begin{matrix*}[l]
			1	&, \abs{\omega }<\omega _c\\
			0	&, \text{ resto}
		\end{matrix*} \right. \]
}
{
	\begin{align*}
		H_D \left( e^{j\omega} \right) & = \biggl. H_C \left( j\omega \right) \biggr\vert_{\omega = \frac{\Omega}{T_s}} = H_C \left( j \frac{\Omega}{T_s} \right) \quad , \abs{\Omega}<\pi \\[10pt]
		H_D \left( e^{j\omega} \right) & = \left\lbrace
		\begin{matrix*}[l]
			1 & , \abs{\frac{\Omega}{T_s} }<\omega _c\\
			0 & , \text{ resto}
		\end{matrix*} \right. = \boxed{\left\lbrace
		\begin{matrix*}[l]
			1 & , \abs{\Omega}<\omega _cT_s<\pi\\
			0 & , \text{ resto}
		\end{matrix*} \right.}
	\end{align*}
Siempre y cuando se cumpla que $\omega_s\geq 2\omega_c$, podré emular en todo el rango de frecuencias de interés $H_c \left( j\omega \right)$
}

\section{Cambio de velocidad de muestreo}

El cambio de velocidad de muestreo (también denominado procesado multitasa) pretende obtener a partir de una señal discreta otra señal discreta con diferente frecuencia de muestreo.

\subsection{Teorema del muestreo discreto}

Para evitar el solapamiento espectral, se debe cumplir: \[ Q \leq \frac{\pi}{\Omega_m}\]

Donde $Q$ es el número de réplicas que caben en el espectro de la señal a muestrear. En otras palabras, $Q$ es el \textbf{orden de diezmado}.

\subsection{Teorema de la interpolación}

Sea $x[n]$ una señal discreta a la que queremos realizar interpolación de orden $P$, la señal interpolada resultante tendrá la forma:
\[ x_P[n] = \left\lbrace 
\begin{matrix*}[l]
	x \left[ \frac{n}{P} \right] & \text{si } n = \dot{P}\\[5pt]
	0 & \text{resto}
\end{matrix*} \right.\]

Lo que se traduce en posicionar $P-1$ ceros entre cada muestra de la señal original.


\section{Aspectos prácticos}

El proceso de muestreo y reconstrucción de una señal nunca es ideal. Se producirán errores, que pueden ser minimizados en función de los parámetros que escojamos en cada proceso.

El \textbf{ADC} agrupa el conjunto de operaciones que nos permite convertir una señal de entrada continua en una señal analógica.

\begin{figure}[ht]
	\centering
	\begin{tikzpicture}[auto,>=latex']
		\node [input, name=input] (input) {};
		\node [block, below of=input, node distance=0cm] (A) {$
		\begin{matrix}
			\text{Filtro}\\
			\text{anti-aliasing}
		\end{matrix}$};
		\node [block, right of=A, node distance=2.5cm] (B) {$
		\begin{matrix}
			\text{Sampler}\\
			\text{\& Hold}
		\end{matrix}$};
		\node [block, right of=B, node distance=2.5cm] (C) {Cuantificador};
		\node [block, right of=C, node distance=2.5cm] (D) {Codificador};

		\draw [color=red,thick, dashed](1.5,-1) rectangle (8.75,1);
		\node[text width=3cm] at (9.25,1.25) {\textbf{ADC}};

		\draw [-angle 90] (-2.5,0) --  node{$x(t)$} (A);
		\draw [-angle 90] (A) --  node{} (B);
		\draw [-angle 90] (B) --  node{} (C);
		\draw [-angle 90] (C) --  node{} (D);
		\draw [-angle 90] (D) --  node{$x[n]$} ++(2.5,0);
	\end{tikzpicture}
	\caption{Etapa de entrada ADC} \label{fig:esquema_ADC}
\end{figure}



\chapter{La Transformada Discreta de Fourier (DFT)}

\section{Introducción}

Hay algunas cosas importantes que debemos recordar sobre la Transformada Discreta de Fourier:

\begin{itemize}
	\item La TF de una señal discreta es \textbf{periódica}, de periodo $2\pi$.
\end{itemize}

\section{Definición, cálculo, relaciones y propiedades}

\subsection{TF para señales periódicas}

\begin{multicols}{2}
	\subsubsection{Ecuación de síntesis}
	
	\[ x[n] = \sum_{k=<N>}^{}a_ke^{jk \frac{2\pi}{N} n}\]
	
	\subsubsection{Ecuación de análisis}
	
	\[ a_k = \frac{1}{N}\sum_{n=<N>}x[n]e^{-jk \frac{2\pi}{N}n} \]
	
\end{multicols}

\subsubsection{DFT}

Sea $x[n]$ una señal periódica de periodo $N$: \[ x[n] = \]

Podemos calcular su TF como:
\[ \TF{x[n]} = X(\Omega ) = 2\pi \sum_{k=<N>}a_k \cdot \delta \left( \Omega - \frac{2\pi}{N}k \right)\]

\subsection{Conversión a secuencia discreta}

La conversión de la DFT de una señal a una secuencia discreta se realiza con la simple relación $\Omega = \omega T_s$, adaptándola según necesitemos. A continuación, la forma de llevarlo a la práctica:

\[ X \left( \omega \right) = \biggl. X \left( \Omega \right)  \biggr\vert _{\Omega = \omega T_s} \qquad \qquad X \left( \Omega \right) = \biggl. X \left( \omega \right)  \biggr\vert _{\omega = \frac{\Omega}{T_s}} \]

\section{Introducción al análisis espectral mediante la DFT}

\section{Filtrado de señales mediante la DFT}

\section{Ejercicios resueltos}


Calcular los coeficientes $a_k$ de la señal indicada a continuación: \[ x[n] = \sen \left( \frac{2\pi}{5}n \right) \]



\chapter{Diseño de filtros digitales} \label{temaDeFiltros}

\section{Introducción}

\section{Diseño de filtros FIR}

\section{Diseño de filtros IIR}

\section{Comparación entre métodos de diseño y tipos de filtros}

\section{Estructuras para la implementación de filtros digitales}

%%% FIN DE LOS APUNTES %%%

\end{document}
