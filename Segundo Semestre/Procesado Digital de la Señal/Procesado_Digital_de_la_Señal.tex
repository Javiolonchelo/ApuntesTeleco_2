\documentclass[a4paper]{book}

\input{../../Archivos comunes/Preamble.tex} % Se incluye el preámbulo

% Título y portada
\title{\Huge Procesado Digital de la Señal\\\vspace*{5pt}
\Large Apuntes de clase}
\author{Javier Rodrigo López \thanks{Correo electrónico: \href{mailto:javiolonchelo@gmail.com}{\texttt{javiolonchelo@gmail.com}}}} 
\date{\today}

%%% INICIO DEL DOCUMENTO %%%
\begin{document}

\setlength{\wpYoffset}{-0.5cm}
\ThisCenterWallPaper{0.7}{./Imágenes/Velazquez.jpg}
\maketitle

% Marca de agua
\AddToShipoutPictureFG{
\begin{tikzpicture}[overlay,remember picture]
\path (current page.south west) -- (current page.north east)
 node[midway,scale=8,color=lightgray,sloped,opacity=0.05] {Javier Rodrigo López};
\end{tikzpicture}
}

% Logotipos UPM y ETSIST
\begin{figure}[t!]
\centering
	\begin{subfigure}[b]{0.65\linewidth}
		\includegraphics[width=\linewidth]{../../Archivos comunes/upm_logo.png}
	\end{subfigure}
	\begin{subfigure}[b]{0.25\linewidth}
		\includegraphics[width=\linewidth]{../../Archivos comunes/etsist_logo.png}
	\end{subfigure}
\end{figure}

% Introducción
\newpage
\phantomsection
\addcontentsline{toc}{section}{Introducción}
\section*{Introducción}
Imagen de la portada: \textsl{La fragua de Vulcano}, por Diego Velázquez.
\newpage

% Índice (TOC)
\setlength{\parskip}{0em}
\tableofcontents 
\setlength{\parskip}{0.5em}

%%% INICIO DE LOS APUNTES %%%
\chapter{Procesado digital de señales continuas}
\section{Introducción}
\section{Muestreo de señales continuas}
\section{Procesado en tiempo discreto de señales continuas}
\section{Procesado en tiempo continuo de señales discretas}
\section{Cambio de velocidad de muestreo}

\chapter{La transformada discreta de Fourier (DFT)}
\section{Introducción}
\section{Definición, cálculo, relaciones y propiedades}
\section{Introducción al análisis espectral mediante la DFT}
\section{Filtrado de señales mediante la DFT}

\chapter{Diseño de filtros}
\section{Introducción}
\section{Diseño de filtros FIR}
\section{Diseño de filtros IIR}
\section{Comparación entre métodos de diseño y tipos de filtros}
\section{Estructuras para la implementación de filtros digitales}
%%% FIN DE LOS APUNTES %%%

%%% BIBLIOGRAFÍA %%%
% Por defecto, se encuentra desactivada. Esto disminuye el tiempo de procesado. Se puede activar cuando se vaya a exportar el PDF definitivo

%\newpage
%\phantomsection
%\label{sec:bibliografia_final}
%\renewcommand{\refname}{Bibliografía}
%\addcontentsline{toc}{section}{Bibliografía}
%\bibliography{bibliografia} % Nombre del archivo (sin ".bib")
%\bibliographystyle{bababbrv}

\end{document}
