\documentclass[a4paper]{book}

\input{../../Archivos comunes/Preamble.tex} % Se incluye el preámbulo

% Título y portada
\title{\Huge Ciencia, Tecnología y Sociedad\\\vspace*{5pt}
\Large Apuntes de clase}
\author{Javier Rodrigo López \thanks{Correo electrónico: \href{mailto:javiolonchelo@gmail.com}{\texttt{javiolonchelo@gmail.com}}}} 
\date{\today}

%%% INICIO DEL DOCUMENTO %%%
\begin{document}

\setlength{\wpYoffset}{0 cm}
\ThisCenterWallPaper{0.7}{./Imágenes/Murillo.jpg}
\maketitle

% Marca de agua
\AddToShipoutPictureFG{
	\begin{tikzpicture}[overlay,remember picture]
		\path (current page.south west) -- (current page.north east)
		node[midway,scale=8,color=lightgray,sloped,opacity=0.05] {Javier Rodrigo López};
	\end{tikzpicture}
}

% Logotipos UPM y ETSIST
\begin{figure}[t!]
	\centering
	\begin{subfigure}[b]{0.65\linewidth}
		\includegraphics[width=\linewidth]{../../Archivos comunes/upm_logo.png}
	\end{subfigure}
	\begin{subfigure}[b]{0.25\linewidth}
		\includegraphics[width=\linewidth]{../../Archivos comunes/etsist_logo.png}
	\end{subfigure}
\end{figure}

% Introducción
\newpage
\phantomsection
\addcontentsline{toc}{section}{Introducción}
\section*{Introducción}
Imagen de la portada: \textsl{Jacob pone las varas al ganado de Labán}, por Bartolomé Esteban Murillo.
\newpage

% Índice (TOC)
\setlength{\parskip}{0em}
\tableofcontents
\setlength{\parskip}{0.5em}

%%% INICIO DE LOS APUNTES %%%
\chapter{Sostenibilidad ecológica y social}

Trabajo:

Rellenar la plantilla de Moodle. Pensar en una inciativa englobada en alguno de los aspectos vistos en el documental Demain. La iniciativa debe ser local, que se desarrolle en tu barrio, en tu escuela o en tu lugar de vacaciones.

\begin{itemize}
	\item En la semana del 22 de marzo, en clase, cada alumno expondrá y defenderá su idea.
	\item Se elegirá la idea a desarrollar, votando en cada grupo.
	\item Se subirán a Moodle las iniciativas de cada alumno, incluyendo la más votada.
	\item A partir de ese momento, se desarrollará el proyecto,
\end{itemize}

\subsubsection{Desarrollo del Trabajo}
\begin{itemize}
	\item Título y estado del arte
	\item Motivación y descripción
	\item Indicadores, datos, obstáculos y presupuesto.
	\item Tecnología de mejora
	\item Conclusión
	\item Otros (referencias, bibliografía...)
\end{itemize}


\section{Cooperación al desarrollo}

\chapter{Las revoluciones tecnológicas}
\section{Política científica}

\chapter{TIC, ética y derechos humanos}
\section{Códigos deontológicos del sector TIC}

\chapter{Casos de uso de tecnología social}

%%% FIN DE LOS APUNTES %%%

%%% BIBLIOGRAFÍA %%%
% Por defecto, se encuentra desactivada. Esto disminuye el tiempo de procesado. Se puede activar cuando se vaya a exportar el PDF definitivo

%\newpage
%\phantomsection
%\label{sec:bibliografia_final}
%\renewcommand{\refname}{Bibliografía}
%\addcontentsline{toc}{section}{Bibliografía}
%\bibliography{bibliografia} % Nombre del archivo (sin ".bib")
%\bibliographystyle{bababbrv} 

\end{document}
