\documentclass[a4paper]{book}

\input{../../Archivos comunes/Preamble.tex} % Se incluye el preámbulo

% Título y portada
\title{\Huge Electrónica II\\
\Large Apuntes de clase}
\author{Javier Rodrigo López \thanks{E-mail: \href{mailto:javiolonchelo@gmail.com}{\texttt{javiolonchelo@gmail.com}}}} 
\date{\today}

%%% INICIO DEL DOCUMENTO %%%
\begin{document}


\setlength{\wpYoffset}{-5 cm}
\ThisCenterWallPaper{0.7}{./Imágenes/El Bosco.jpg}
\maketitle


% Marca de agua
\AddToShipoutPictureFG{
	\begin{tikzpicture}[overlay,remember picture]
		\path (current page.south west) -- (current page.north east)
		node[midway,scale=8,color=lightgray,sloped,opacity=0.05] {Javier Rodrigo López};
	\end{tikzpicture}
}

% Logotipos UPM y ETSIST
\begin{figure}[t!]
	\centering
	\begin{subfigure}[b]{0.65\linewidth}
		\includegraphics[width=\linewidth]{../../Archivos comunes/upm_logo.png}
	\end{subfigure}
	\begin{subfigure}[b]{0.25\linewidth}
		\includegraphics[width=\linewidth]{../../Archivos comunes/etsist_logo.png}
	\end{subfigure}
\end{figure}


\newpage

\phantomsection
\setlength{\parskip}{0.5em}

\addcontentsline{toc}{section}{Introducción}
\section*{Introducción}
Esta pequeña recopilación de fórmulas, teoremas y demás apuntes de teoría ha sido elaborada durante el primer semestre del curso 2019-2020, en la escuela \href{https://www.etsist.upm.es/}{\textbf{ETSIST}} de la \href{http://www.upm.es/}{\textbf{UPM}} por Javier Rodrigo López, alumno de 2º curso de Ingeniería de Sonido e Imagen.

Muchas gracias al profesor \href{https://www.euitt.upm.es/escuela/directorio?departamento=DTE&idTrabajador=c0874849914b0380a54a53f80d4d60f1}{\textbf{José Antonio Herrera Camacho}} por hacer una labor excelente como docente y despertar en mí el interés por la Electrónica. Una gran parte de estos apuntes existiría si no hubiera asistido a sus clases.
\newpage

\setlength{\parskip}{0em}
\tableofcontents
\setlength{\parskip}{0.5em}


\chapter{Bloque Temático 1}

\section{Codificación de la información}

Para convertir una señal analógica en digital, tenemos que aplicarle una transformación denominada \textbf{codificación}.

La codificación
\section{Codificación de números}
\subsection{Código BN (Binario Natural)}

La fórmula es \[n\geq \log_2{m}\]
Siendo $n$ el número de bits y $m$ el número de objetos. Aunque la mayoría de las veces

El rango de números naturales que se pueden codificar con $n$ bits es:
\[\left( 0, 2^{n}-1 \right)\]

Existen distintas opciones de asignación de códigos. \begin{itemize}
	\item A cada elemento del conjunto debe corresponderle un único código de 0s y 1s.
	\item Se debe utilizar el \textbf{menor número de bits} posible.
	\item La asignación debe favorecer las operaciones de codificar y descodificar.
\end{itemize}
\subsection{Código BCD (Binary Coded Decimal)}
Representa cada una de las cifras decimales de un número expresado en decimal con 4 bits.

Un número decimal de $n$ cifras necesita $4n$ bits para expresarlo en BCD.

El BCD se utiliza mucho para interfaces con personas, ya que cada \textit{nibble} representa
\section{Aritmética Binaria}

\subsection{Suma de números}
De toda la vida. Aquí van algunos ejemplos...

Puede sobrepasar el número de bits con el que se trabaja y se produce un \textbf{acarreo} o \textbf{carry}.

\subsection{Resta de números}
De toda la vida. Aquí van algunos ejemplos...

Puede producirse un error por no utilizar suficientes bits y que el resultado no sea fiel a la operación. A esto se le denomina \textbf{overflow}.

\section{Ejercicios sobre codificación y Aritmética Binaria}


\section{Álgebra de Boole}
\subsection{Teoremas básicos}

\begin{itemize}
	\item \textbf{Teorema de idempotencia}:
	      \begin{multicols}{2}
		      \[\boxed{a + a = a }\]
		      \[\boxed{ a \cdot  a = a}\]
	      \end{multicols}
	\item \textbf{Teorema de absorción}:
	      \begin{multicols}{2}
		      \[\boxed{a + ab = a}\]
		      \[\boxed{ a \cdot  (a+b) = a}\]
	      \end{multicols}
	\item \textbf{Teorema de adyacencia}:
	      \begin{multicols}{2}
		      \[\boxed{ab + a\overline{b} = a}\]
		      \[\boxed{ (a+b)(a+\overline{b}) = a}\]
	      \end{multicols}
	\item \textbf{Teorema de simplificación}:
	      \begin{multicols}{2}
		      \[\boxed{a + \overline{a}b = a + b}\]
		      \[\boxed{ a (\overline{a}+b) = ab}\]
	      \end{multicols}
	\item \textbf{Teorema de De Morgan}:
	      \begin{multicols}{2}
		      \[\boxed{\overline{a_0 \cdot a_1\cdot \ldots \cdot a_n} = \overline{a_0} + \overline{a_1} + \ldots + \overline{a_n}}\]
		      \[\boxed{ \overline{a_0 + a_1+ \ldots + a_n} = \overline{a_0} \cdot  \overline{a_1} \cdot  \ldots \cdot  \overline{a_n} }\]
	      \end{multicols}
\end{itemize}

\section{Cronogramas}
Un cronograma es un diagrama donde se representan las entradas, nodos intermedios y salidas en función del tiempo.

Para esta parte del temario, y a falta de laboratorio en esta asignatura, intentaremos utilizar la aplicación Quartus. Esta herramienta CAD nos permitirá trabajar con puertas lógicas.


\section{Sistemas combinacionales complejos}
La complejidad de un circuito combinacional depende sobre todo del \textbf{número de entradas}, ya que con $N$ entradas la tabla de verdad tendrá $2^N$ filas.

Hemos encontrado casos en los que los circuitos tenían muy poquitos unos o muy poquitos ceros en su tabla de verdad, o que se podían escribir de forma resumida.

\subsection{Divide y vencerás}
La estrategia principal consiste en dividir el circuito en otros más pequeños que, al agruparlos, sean fácilmente descriptibles.

Por ejemplo, un sumador de $N$ bits. Podemos hacer un sumador de 1 bit y con

\chapter{Bloque Temático 2}
\section{Arquitecturas digitales I}
\subsection{Sistemas basados en microprocesador}
\subsection{Sistemas cableados}
Pros: \begin{itemize}
	\item Velocidad y eficiencia.
\end{itemize}
Contras: \begin{itemize}
	\item Complejidad del hardware.
	\item Versatilidad.
\end{itemize}

\section{Conceptos básicos}
\subsection{Nivel logico}
La representación de ceros y unos se define por intervalos de tensiones.

Es decir, si el nivel alto se corresponde con una tensión $V_{CC}$

\subsection{Células lógicas}
Son circuitos electrónicos que realizan operaciones lógicas.

\subsubsection{Nodos lógicos}
Un nodo lógico es una interconexión entre células lógicas.

\subsection{Modelo lógico de las células}
Células estándar: Función lógica prefijada en el diseño de la célula. Por ejemplo, una NOR de dos entradas, construida con transistores CMOS.


\section{Tecnologías}
\subsection{Circuitos cableados}
\subsubsection{SSI. Ciruitos de baja escala de integracíón}
Tecnología de los años 60. Métodos manuales de diseño. Aplicación marginal. Lógica auxiliar (Glue logic)
\subsubsection{Ciruitos de media escala de integracíón}
MSI.
Como mucho, 100 transistores.
\subsubsection{ASIC. Ciruitos de media escala de integracíón}
Tecnología VLSI (Very Large Scale Integration).

\subsubsection{Lógica configurable}
Tecnología VLSI.
Funcionamiento lógico configurable mediante la descarga de un fichero.
\section{Arquitecturas digitales II}

\subsection{Estructura de un sistema digital}

PCB (Printed Circuit Board). Es básicamente la placa donde se colocan los componentes

Alimentación
Interfaces
Disipadores

\section{Tecnología II}
\subsection{Introducción al modelado de CIs}
Necesitamos modelos simplificados para abordar el diseño de sistemas muy complejos.
Hay tres modelos:
\begin{enumerate}
	\item Lógico. Función lógica e interfaz.
	\item Eléctrico. Características eléctricas de la interfaz.
	\item Dinámico.
\end{enumerate}
\subsubsection{Modelo lógico}
Decribe el funcionamiento y la interfaz del chip.
\subsubsection{Modelo eléctrico}
Es muy importante para conectarlo

\chapter{Bloque Temático 3}


\section{Introducción a los circuitos secuenciales}
\section{Cronogramas funcionales de circuitos de flip-flops}
\section{Registros}
\section{Diseño de autómatas}
Un \textbf{autómata de Moore} es un autómata en el que la salida depende únicamente del estado actual de la memoria.

Diagrama de estados para autómatas de Moore:

Un \textbf{autómata de Mealy} es un autómata en el que la salida depende del estado actual de la memoria y de las entradas.

Diagrama de estados para autómatas de Mealy:

\section{Contadores}
\section{Metodología completa de diseño de sistemas}




% BIBLIOGRAFÍA
%\newpage
%\phantomsection
%\label{sec:bibliografia_final}
%\renewcommand{\refname}{Bibliografía} % Para cambiar el título al incluir una librería de BibTeX
%\addcontentsline{toc}{section}{Bibliografía}
%\bibliography{biblio_Tex}
%\bibliographystyle{bababbrv} 

\end{document}
